\section{Introduction}\label{sec:introduction}

The semantic property of projection, traditionally associated with
presuppositions, has challenged many structure-driven formal semantic
analyses. Linguistic content is said to project if it is interpreted outside
the scope of an operator that syntactically subordinates it. This happens,
for example, if a presupposition-triggering expression is introduced within
the scope of a negation: in the sentence ``\textit{It is not the case that
the man is happy}'' the definite noun phrase ``\textit{the man}''
syntactically occurs inside the scope of the negation, but it is interpreted
as if occurring outside of its scope (corresponding to the reading: $\exists
x(man(x) \wedge \neg happy(x))$). In semantic formalisms, this behaviour has
often been treated as a deviation from standard meaning construction,
despite the prevalence of expressions exhibiting it
\citeaffixed{sandt1992presupposition,heim1983projection,zeevat1992presupposition,geurts1999presuppositions,beaver2001presupposition}{see,
e.g.,}.  Therefore, we have proposed a formalism that centralizes the
property of projection as a strategy for integrating material into the
foregoing context. This formalism is called Projective Discourse
Representation Theory \citeaffixed{venhuizen2013iwcs}{PDRT;}, and is an
extension of the widely applied framework Discourse Representation Theory
\cite{kamp1981theory,kamp1993discourse}. In PDRT, presuppositions are
treated as part of a larger class of \textit{projection phenomena}
\citeaffixed{simons2010projects}{cf.}, including anaphora and conventional
implicatures \citeaffixed{potts2005logic}{cf.}. By associating linguistic
material with a \textit{pointer} to indicate its interpretation site, an
explicit distinction is made between the surface form of an utterance, and
its logical interpretation. This analysis can be shown to account for
various projection phenomena, including presuppositions
\cite{venhuizen2013iwcs} and conventional implicatures
\cite{venhuizen2014salt}. Crucially, however, adding projection
pointers to all linguistic material affects the formal properties of DRT
non-trivially; the occurrence of projected material at the interpretation
site results in non-hierarchical variable binding, and violates the
traditional DRT notion of accessible contexts, thereby compromising the
basic construction procedure.

In this paper, we will describe the formal implications of extending DRT
with projection pointers. After briefly describing the motivation behind
this DRT extension, we systematically derive the formal definitions of PDRT
from the DRT definitions for building and combining structures, and show
where they diverge (Sections~\ref{sec:building} and~\ref{sec:combining}). By
providing a translation from PDRT to DRT in Section~\ref{sec:translating},
we show that PDRT inherits all formal interpretive properties of DRT,
together with its inference mechanisms.  We introduce an implementation of
DRT and PDRT, called \textsc{pdrt-sandbox}, which implements all these
definitions, and provides a full-fledged toolkit for use in NLP
applications. In Section~\ref{sec:playing}, we offer some directions for
applying the formalism and its implementation. Finally,
Section~\ref{sec:conclusions} concludes the paper.

\subsection{Discourse Representation Theory}

In the dynamic framework of Discourse Representation Theory
\citeaffixed{kamp1993discourse}{DRT;}, the meaning of a discourse is
represented by means of recursive structures called \textit{Discourse
Representation Structures} (DRSs). A DRS consists of a set of
\textit{discourse referents} and a set of \textit{conditions} on these
referents. Conditions may be either \textit{basic}, reflecting a property or
a relation between referents, or \textit{complex}, reflecting logical
structure introduced by semantic operators such as negation, implication,
and modal operators. DRSs are often visualized using a box-representation
consisting of two parts: the set of referents is represented in the top of
the box, and the set of conditions on these referents are shown in the body.
Example \Next shows the DRS of a donkey sentence
\citeaffixed{geach1962reference}{adapted from}, containing a single complex
condition representing an implication, which itself consists of two embedded
DRSs, the first containing two referents and a set of three basic
conditions, and the second containing a single basic condition.

\ex. If a farmer owns a donkey, he feeds it.\\\hspace*{-0.2cm}
  \drs{}{
    \drs{x~~y}{farmer(x)\\ donkey(y)\\ own(x,y)}
    $\Rightarrow$
    \drs{}{feed(x,y)}}

Each (embedded) DRS can be seen as representing a \textit{context},
together reflecting the logical form of the discourse. Crucially, anaphoric
binding of referents is determined on the basis of the accessibility between
contexts; in the example above, the antecedent DRS of the implication is
accessible from the consequent DRS, and as a result the variables introduced
by the pronouns (``\textit{he}'' and ``\textit{it}'') become bound by the
referents introduced in the antecedent DRS. 

The straightforward analysis of anaphora was one of the main motivations
behind the development of DRT \cite{kamp1981theory,heim1982semantics}, but
it has been shown that the framework can account for a wide range of other
linguistic phenomena as well, including: tense \cite{kamp1981theory},
quantification and plurality \cite{kamp1993discourse}, attitude reports
\cite{asher1986belief,asher1989belief,zeevat1996neoclassical,maier2009presupposing}
discourse structure \cite{asher2003logics}, and presupposition
\cite{sandt1992presupposition,krahmer1998presupposition,geurts1999presuppositions}.
While this illustrates the flexibility the framework, the analysis of
presuppositions, in particular, also points out some limitations of the
framework. Next, we will describe these limitations and motivate the
proposed extension to DRT: Projective DRT.

\subsection{Beyond the surface structure}

In the DRT framework, \textit{form} determines \textit{interpretation}. That
is, the context in which some content is introduced determines how it is
interpreted. Thus, if some content occurs within the (syntactic) scope of,
for example, an implication or a negation in the linguistic surface form, it
will be interpreted within the logical scope of this operator. This tight
correspondence between form and meaning challenges a parsimonious account of
presuppositions and other \textit{projection phenomena}
\citeaffixed{potts2005logic}{e.g., conventional implicatures; cf.}, as these
are generally interpreted outside the logical scope of any operator that
embeds them \citeaffixed[for an overview]{simons2010projects}{see}.
Consider the following example, which is identical to \Last except for the
replacement of the indefinite noun phrase `a farmer' by the proper name
`John', triggering the presupposition that there exists some person
named \textit{John}.

\ex. If John owns a donkey, he feeds it.

In order to obtain the desired analysis for this sentence, it does not
suffice to simply replace the predicate in the associated DRS, since this
results in a reading that states that if there exists a person named John
who owns a donkey, then he feeds it.  In the desired interpretation,
however, the presupposition about the existence of `John' should be
moved to outside the scope of the implication, resulting in a reading in
which there exists some person named \textit{John}, for whom it holds that if he
owns a donkey, then he feeds it.  

This process of resolving presuppositions in DRT was formalized by
\citeasnoun{sandt1992presupposition}. In the account that has become known
as `binding and accommodation theory'
\cite{geurts1999presuppositions,beaver2002presupposition,bos2003implementing},
\citename{sandt1992presupposition} treats presupposition projection as
a variety of anaphora resolution: a presupposition first occurs marked at
its introduction site, and after discourse construction it is either
\textit{bound} to an accessible antecedent, or \textit{accommodated} at an
accessible context. This is illustrated below, where the presupposition
triggered by the proper name `John' yet needs to be resolved in Stage
I (indicated using a dashed box) \Next[a], and has been accommodated to the
global context in Stage II \Next[b].

\begin{flushleft}
\begin{minipage}{0.85\linewidth}
\begin{multicols}{2}
\ex.\a.Stage I\\\hspace*{-0.2cm}{
  \drs{}{
    \drs{y}{donkey(y)\\ own(x,y)\\ \begin{tabular}{l}\noalign{\smallskip}
        \begin{tabular}{:l:} \hdashline x\\\hdashline John(x)\\\hdashline 
       \end{tabular}\smallskip\end{tabular}}
    $\Rightarrow$
    \drs{}{feed(x,y)}}
  }
  \b.\vspace*{-0.4cm}Stage II\\\hspace*{-0.2cm}{
  \drs{x}{John(x)\\
    \drs{y}{donkey(y)\\ own(x,y)}
    $\Rightarrow$
    \drs{}{feed(x,y)}}
  }

\end{multicols}
\end{minipage}\\
\end{flushleft}

\noindent Although the algorithm proposed by binding and accommodation
theory results in representations with the desired interpretation for
presuppositions, the process of arriving at this interpretation, as well as
the obtained representations, are not completely satisfactory. Firstly, the
two-stage process of deriving the representations is at odds with
a compositional construction procedure for DRSs, as presuppositions are only
resolved \textit{after} discourse construction, and the intermediate
representations with unresolved DRSs are ill-defined with respect to formal
properties like accessibility and variable binding. Secondly, since in the
final representations no distinction can be made between asserted and
presupposed content, the information structure of the discourse becomes
obliterated \citeaffixed{kracht1994logic,krahmer1998presupposition}{cf.}.

%, and distinction between asserted and
%projected content (i.e., explicit information structure)

In order to provide a parsimonious account of presuppositions, as part of
the larger class of projection phenomena, we have proposed an extension to
DRT, called Projective Discourse Representation Theory
\citeaffixed{venhuizen2013iwcs}{PDRT;}, in which an explicit distinction is
made between the context in which some linguistic content is
\textit{introduced}, and where it is \textit{interpreted}, reflecting the
separation between linguistic surface form and truth-conditional
interpretation.

\subsection{Projective Discourse Representation Theory}

In Projective Discourse Representation Theory, the basic structures carry
more information than the basic DRT structures; in addition to the
structural and referential content of a DRS, a Projective Discourse
Representation Structure (PDRS) also makes the information structure of
a discourse explicit by keeping linguistic content at its introduction site,
and indicating the interpretation site via a variable. 
Each PDRS introduces
a \textit{label} that can be used as an identifier
\citeaffixed{asher2003logics}{similar to the context identifiers implicitly
assumed by other DRT extensions, for example SDRT;}.  All referents and
conditions of a PDRS are associated with a \textit{pointer}, which is used
to indicate in which context the material is \textit{interpreted} by means
of binding the pointer to a context label.  Example \Next shows two PDRSs:
here, the labels introduced by the PDRSs are shown on top of each PDRS, and
the pointers associated with referents and conditions are indicated using
a leftward pointing arrow ($\gets$). Note that each PDRS also contains a footer for
describing accessibility relations between contexts; these are called
Minimally Accessible Projection contexts (MAPs), and will be motivated
below.

\begin{flushleft}
\begin{minipage}{0.9\linewidth}
  \begin{multicols}{2}
\ex. \a. Nobody sees a~man.\\\hspace*{-0.3cm}{
  \pdrs{1}{}{
    $1\gets\neg$\pdrs{2}{$2\gets x, 2\gets y$}{
      $2\gets$person($x$)\\ $2\gets$man($y$)\\ $2\gets$see($x,y$)
    }{}
  }{}
}
\b. Nobody sees John.\\\hspace*{-0.3cm}{
  \pdrs{1}{}{
    $1\gets\neg$\pdrs{2}{$2\gets x, 5\gets y$}{
      $2\gets$person($x$)\\ $5\gets$John($y$)\\ $2\gets$see($x,y$)
    }{$2 \leq 5$}
  }{}
}

\end{multicols}
\end{minipage}\\
\end{flushleft}

\noindent In \Last[a], all pointers are bound by the label of the PDRS in
which the content is introduced, indicating asserted material which is
interpreted locally (corresponding to the non-specific reading of the
indefinite noun phrase ``a man'').  This representation is identical to the
standard DRT representation of this sentence, except for the addition of
labels to PDRSs and pointers to all referents and conditions. In \Last[b],
on the other hand, the proper name ``John'' triggers a presupposition about
the existence of someone called `John'.  The pointer associated with the
referent and condition describing this presupposition occurs free (i.e., it
is not bound by the label of any accessible PDRS), meaning that no
antecedent has been found yet; the pointer may still become bound in some
higher context during discourse construction, which means that the content
will be interpreted in that context, or it may remain free, which means that
the content needs to be accommodated; this accommodation site may be
determined based on various heuristics, for example the pragmatic
constraints proposed by \cite{sandt1992presupposition}.

An important addition to the representation structures in PDRT is the
introduction of \textit{Minimally Accessible Projection contexts} (MAPs),
which where not part of the original PDRT description in
\citeasnoun{venhuizen2013iwcs}. As the name suggests, MAPs pose minimal
constraints on the accessibility of projection contexts, which may either be
contexts available as PDRSs in the global PDRS, or projected contexts
created by free pointers.  Basically, the MAPs are pairs of subordinating
contexts resulting in a partial order over PDRS contexts, similar to the
partial order over underspecified DRSs described by
\citename{reyle1993dealing} \citeyear{reyle1993dealing,reyle1995reasoning}.
With this additional level of information, constraints on (projection)
contexts can be made explicit in the current framework, such as the
constraint that the projection site of presuppositions should be accessible
from their introduction site. As we will see below (in
Section~\ref{sec:playing}) these MAPs also allow for a principled definition
of the projection pattern of embedded presuppositions and conventional
implicatures \citeaffixed{venhuizen2014salt}{cf.}

As described above, one of the main assets of the DRT framework is its
treatment of anaphora, which crucially depends on a systematic definition of
context accessibility and binding of referents. The addition of pointers and
labels to PDRSs affects these definitions non-trivially, since projected
content now appears \textit{in situ}, but still keeps the binding properties
from its interpretation site. To see this, consider the following example:

\ex. It is not the case that John is a vegetarian, he's a vegan.\\\hspace*{-0.3cm}{
  \pdrs{1}{}{
    $1\gets\neg$\pdrs{2}{$3\gets x$}{
      $3\gets$John($x$)\\ $2\gets$vegetarian($x$)
        }{$2 \leq 3$}\\
      $1\gets$vegan($x$)
  }{$1\leq 3$}
}

In this PDRS, the variable $x$ in the predicate \textit{vegan} is bound by
the discourse referent introduced by `John', despite the fact that it occurs
in the universe of an embedded (non-accessible) PDRS. Since the referent is
associated with a free pointer, indicating a presupposition, it can bind
variables that occur in PDRSs that are not accessible from the PDRS in which
it occurs; the only requirement is that the context indicated by the pointer
of the referent is accessible from the context indicated by the pointer of
the predicate in which the variable occurs. Again, the set of MAPs can be
used to make this requirement explicit; $1\leq 3$ here indicates that
context $3$ should be accessible from the PDRS labelled $1$, which happens
to be the context to which the referent introduced by `John' projects.  In
the next section, we will formalize these accessibility constraints for DRT
and PDRT, after providing a formal definition of DRSs and PDRSs. Based on
these definitions, we can then formally describe the compositional and
interpretational properties of (Projective) Discourse Representation
Structures.

\subsection{Terminology and Notation}

Before going into a description of the formal definitions for DRT and PDRT,
let us provide some basic handles for the terminology and notation used
throughout this paper. 

[...]
%%XXX
