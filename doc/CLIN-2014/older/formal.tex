\section{Formalizing Projection}

Formal analyses of projection phenomena have mainly focused on providing
a formal account of presuppositions, as these are considered the most
paradigmatic cases of projection.  In order to account for projection
behaviour, a semantic theory needs to be able to account for the interaction
between the projected content and the context in which it is introduced.
This means that the framework should be \emph{dynamic}, in the sense that it
can model how pieces of discourse can change the existing context. Various
such frameworks have been introduced, such as Dynamic Predicate Logic
\citep[DPL;][]{groenendijk1991dynamic}, Update semantics
\citep{veltman1991defaults,veltman1996defaults} and Discourse Representation
Theory \citep[DRT;][]{kamp1981theory,kamp1993discourse}. In fact, each of
these frameworks -- or some modification thereof -- has been applied as
a formalism for presupposition; most notably, \cites{heim1983projection}
treatment of presuppositions in File Change Semantics, the `presupposition
as anaphora' account of \citet{sandt1992presupposition}, formalized in DRT,
\cites{zeevat1992presupposition} treatment of presupposition in update
semantics, and \cites{beaver2001presupposition} dynamic account of
presupposition.  

The goal of the current paper is to provide a unified
analysis of different projection phenomena that accounts for presuppositions
as well as conventional implicatures and anaphoric expressions
\citep[cf.][]{simons2010projects}. To achieve this, we need a formal theory
that can treat the different phenomena using a single mechanism for
projection. The obvious candidate for this endeavor is Discourse
Representation Theory (DRT), as this framework was specifically developed to
treat (cross-sentential) anaphoric relations \citep{kamp1993discourse} and
has been shown to provide a fertile setting for treating presuppositions
\citep{sandt1992presupposition,geurts1999presuppositions}. Below, we will
describe the advantages of the DRT approach and show on what points it might
be improved to provide a more general formalism for projection.  First we
look at projection in more detail and describe the main classes of
projection phenomena: presuppositions, anaphora and conventional
implicatures.

[...]

% mention other approaches? Eg Potts multidimensional semantics etc?


\subsection{Presuppositions as anaphora}

One of the most influential treatments of presuppositions is the
`presuppositions as anaphora' approach, first presented in
\citet{sandt1989presupposition} and later worked out in
\citet{sandt1992presupposition} and \citet{geurts1999presuppositions} On
this view, presuppositions behave like anaphora in that they can
anaphorically bind to an earlier introduced antecedent \citep[see
also][]{soames1979projection,kripke2009presupposition}.  Unlike anaphora,
however, presuppositions can occur felicitously in contexts where no
suitable antecedent can be found by creating their own antecedent at an
accessible discourse level; this is called \emph{accommodation} (first
described by \citealp{karttunen1974presupposition}, and
\citealp{stalnaker1974pragmatic}, but named as such by
\citealp{lewis1979scorekeeping}).


In order to formalize this behaviour, \citet{sandt1992presupposition}
assumes that in the construction of a discourse representation
presuppositional content is resolved via a two-stage process. First,
a presupposition introduces marked content at its introduction site (in what
is called an \emph{A-structure}). Only after the representation of the
entire discourse is constructed, the presuppositions are resolved by either
binding them to an available antecedent, or accommodating them at a suitable
level of discourse. An example of the working of this algorithm is shown in
\Next, where \Next[a] shows the unresolved semantics with the
presuppositions occurring at the projection site (the $A$-structure is
indicated as a dashed box), and \Next[b] shows the final DRS with the
presupposed content moved to its accommodation site.  

\begin{multicols}{2}
\ex. It's not the case that John walks.
\a. Unresolved:\\\hspace*{-0.2cm}{\small\drs{}{$\neg$~\drs{}{
      \begin{tabular}{l}\noalign{\smallskip}
        \begin{tabular}{:l:} \hdashline x\\\hdashline John(x)\\\hdashline 
       \end{tabular}\smallskip\end{tabular}\\ walk(x)}}}
  \columnbreak
  \b. \vspace*{0.4cm} Resolved:\\\hspace*{-0.2cm}{\small \drs{x}{John(x)\\ 
    $\neg$~\drs{}{walk(x)}}}

\end{multicols}

\noindent The two-stage resolution algorithm secures projection and binding of
presuppositions based on several constraints that determine relative
preferences between alternative interpretations. These constraints include,
for example, that binding is preferred over accommodation, and that global
accommodation is preferred of local accommodation. This latter constraint
implies that presuppositions will always try to create an antecedent in the
highest accessible context, i.e., outscoping as many operators as possible.
For example, in \Next,
%\citep{beaver2011presupposition}, % example (33b)
in addition to the global context ($c_0$), the modal \emph{maybe} creates an
embedded context ($c_1$), as does the attitude verb \emph{think}~($c_2$).

\ex. ($c_0$) Maybe ($c_1$) Wilma thinks that ($c_2$) her husband is having
  an affair.

The presupposition that Wilma is married, which is triggered by the
possessive construction \emph{her husband}, is not entailed by any of the
contexts and thus needs to be accommodated. Local accommodation in $c_2$,
results in a reading in which maybe Wilma thinks that she is married and her
husband is having an affair, whereas intermediate accommodation in $c_1$
results in a reading in which maybe Wilma is married and thinks that her
husband is having an affair. The preferred reading, however, results from
global accommodation of the presupposition, and states that Wilma is married
and maybe she thinks that her husband is having an affair.

%  \citet{bos2003implementing}

\subsection{Evaluating the DRT approach}

The `presupposition as anaphora' account by \citet{sandt1992presupposition}
and \citet{geurts1999presuppositions} has been very influential in the
development of presupposition theory, but it has also received much
criticism \citep[see, e.g.,][]{beaver2002presupposition}. First of all, the
method of moving semantic material to outside the scope of semantic
operators is not entirely satisfactory, because it reduces the strong
correspondence between the DRS representation and the surface form of an
utterance.  Moreover, in the final representation of
\citet{sandt1992presupposition}, accommodated presuppositions and asserted
content are indistinguishable. This has various disadvantages, as argued by
\citet{kracht1994logic}. Firstly, accommodated presuppositions and asserted
content affect the truth conditions of their context differently, since
falsehood of a presupposition (\emph{presupposition failure}) makes the
sentence in which it occurs undefined, while in the case of falsely asserted
content, the sentence is simply false. Secondly, if we take a compositional
approach to semantics, accommodated presuppositions may become bound later
on, when more information is available. Therefore, accommodated
presuppositions and asserted content should remain distinguishable after
presupposition resolution. In order to resolve this isse,
\citet{krahmer1998presupposition} introduces an extension to
\citeauthor{sandt1992presupposition}'s DRT approach, using a framework
called Presuppositional DRT. In this framework, presuppositions keep their
`presuppositionhood' by means of a special marker for accommodated
presuppositions in the final representation. Although this new feature
increases the compositionality of the analysis, the two-stage resolution
strategy remains, making it not purely compositional.  Moreover, the account
proposed by \citet{krahmer1998presupposition} is not easily generalizable to
other projection phenomena, such as \cites{potts2005logic} conventional
implicatures.

Another challenge for van der Sandtian approaches is to account for factive
presuppositions. In factive constructions, the argument of the factive verb
(e.g. `to know', `to realise', 'to regret') is taken to be presupposed. The
difficulty for traditional approaches to presupposition is that the
presupposed material cannot simply be moved to the global context, since the
argument of the verb would become empty. A solution is to copy the semantic
material, which secures that it is both projected and part of the embedded
context. However, this results in much redundancy and hardly readable
representations.

In sum, despite its empirical coverage, \cites{sandt1992presupposition}
account of presupposition fails to generalize the behaviour of the different
projection phenomena, and assumes a two-step resolution algorithm, resulting
in an analysis that is not purely compositional. In the next section, we
will describe an extension of DRT, called Projective DRT, in which
projection is treated using variable binding.
%%XXX
