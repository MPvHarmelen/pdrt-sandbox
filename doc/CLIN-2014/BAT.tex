% Binding and Accommodation Theory

\subsection{Presuppositions as anaphora}

One of the most influential treatments of presuppositions is the
`presuppositions as anaphora' approach, first presented in
\citeasnoun{sandt1989presupposition} and later worked out in
\citeasnoun{sandt1992presupposition} and
\citeasnoun{geurts1999presuppositions}. On this view, presuppositions behave
like anaphora in that they can anaphorically bind to an earlier introduced
antecedent \citeaffixed{soames1979projection,kripke2009presupposition}{see
also}. Unlike anaphora, however, presuppositions can occur felicitously in
contexts where no suitable antecedent can be found by creating their own
antecedent at an accessible discourse level; this is called
\emph{accommodation} (first described by
\citeasnoun{karttunen1974presupposition}, and
\citeasnoun{stalnaker1974pragmatic}, but named as such by
\citeasnoun{lewis1979scorekeeping}).

The `presupposition as anaphora' account introduced by
\citeasnoun{sandt1992presupposition} is formalized in the framework of
Discourse Representation Theory.
%%XXX introduce DRT here?
On this account, presuppositional content is resolved during the
construction of a discourse representation via a two-stage process. First,
a presupposition introduces marked content at its introduction site (in what
is called an \emph{A-structure}). Then, after the representation of the
entire discourse is constructed, the presuppositions are resolved by either
binding them to an available antecedent, or accommodating them at a suitable
level of discourse. An example of the working of this algorithm is shown in
\Next, where \Next[a] shows the unresolved semantics with the
presuppositions occurring at the introduction site (the $A$-structure is
indicated as a dashed box), and \Next[b] shows the final DRS with the
presupposed content moved to its accommodation site.  

\begin{multicols}{2}
\ex. Every woman loves Mark. %Darcy, Tuitert
\a. Unresolved:\\\hspace*{-0.2cm}{\small\drs{}{\drs{x}{
       woman(x)} $\Rightarrow$ \drs{}{loves(x,y)\\ \begin{tabular}{l}\noalign{\smallskip}
        \begin{tabular}{:l:} \hdashline y\\\hdashline Mark(y)\\\hdashline 
       \end{tabular}\smallskip\end{tabular}}}
       }
  \columnbreak
  \b. \vspace*{0.1cm} Resolved:\\\hspace*{-0.2cm}{\small \drs{y}{Mark(y)\\ 
    \drs{x}{woman(x)}$\Rightarrow$\drs{}{loves(x,y)}}}

\end{multicols}

\noindent The two-stage resolution algorithm secures projection and binding
of presuppositions based on several constraints that determine relative
preferences between alternative interpretations. These constraints include,
for example, that binding is preferred over accommodation, and that global
accommodation is preferred of local accommodation. This latter constraint
implies that presuppositions will always try to create an antecedent in the
highest accessible context, i.e., outscoping as many operators as possible.
In example~\Last above, the embedded structure created by the implication
introduces two possible accommodation contexts in addition to the global
context ($c_0$); the antecedent of the implication ($c_1$), and the
consequent of the implication ($c_2$). The presupposition triggered by the
proper name ``\textit{Mark}'' is not entailed by any of the accessible
contexts, which means that it needs to be accommodated. Local accommodation
in $c_2$ results in a reading in which for every woman, there exists some
person named Mark whom she loves, whereas intermediate accommodation in
$c_1$ results in a reading in which if both a woman and a person named Mark
exists, she loves him. The preferred reading (without any further context),
however, results from global accommodation of the presupposition, and states
that there exists some person named Mark of whom it holds that every woman
loves him.

The treatment of presupposition projection as anaphora resolution offers an
intuitive and empirically adequate account of presuppositions. However, the
two-stage process for resolving presuppositions separates them from
`ordinary' interpretation, in that they require the discourse structure to
be completely constructed before the content can be interpreted. This is not
only problematic from a compositional perspective, but also precludes
a complementary treatment of other projection phenomena, such as
conventional implicatures, as this would require another post-hoc processing
step. Moreover, the fact that presuppositions, once resolved, are
indistinguishable from asserted content does not fare well with the idea
that presuppositional content has a status that is somehow prior to asserted
content \cite{kracht1994logic,krahmer1998presupposition}. The framework of
Discourse Representation Theory seems insufficient for representing this
kind of information structure, as it only reflects the logical structural
dependencies. \citeasnoun{krahmer1998presupposition} aims to resolve this by
describing Presuppositional DRT, which introduces a special marker for
presuppositional content. However, this again places presuppositions on
a special pedestal and makes it difficult to generalize the analysis to
other projection phenomena. A compositional, informative and unified
treatment of different projection phenomena thus seems to require a more
expressive formalism in which projection is inherently part of discourse
interpretation. In this light, we introduce Projective Discourse
Representation Theory, a formalism that centralizes projection by making an
explicit distinction between the introduction and interpretation site of all
linguistic content.

