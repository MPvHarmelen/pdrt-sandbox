\section{Syntax, subordination and binding}
\label{sec:building}

In this section, we describe the basic properties of (Projective) Discourse
Representation Structures in terms of their syntax and accessibility
relations. In this section and throughout the rest of the paper, we will
describe the formal properties of DRSs \citeaffixed[and references described
therein]{kamp1993discourse,bos2003implementing,kamp2011discourse}{variations
of which can also be found in, e.g.,}, and accordingly we derive these same
properties and definitions for PDRSs.

%syntax, subordination & accessibility, variable binding

\subsection{Basic Structures}

The box representations shown in Section~\ref{sec:introduction} are an
intuitive way to look at DRSs and PDRSs, but less useful for formal
reasoning about these structures. For this, we adhere to the set-theoretical
underpinnings of the formalism as described by
\citeasnoun{kamp1993discourse}, without their `duplex conditions', which do
not allow for a direct translation to first-order logic (which we will
provide in Section~\ref{sec:translating}). This definition basically follows
the one proposed by \citeasnoun{bos2003implementing}, with the exception
that equality between variables (i.e., $x_1=x_2$) is not a separate DRS
condition, but is treated as a variant of a 2-place predicate (i.e.,
$R(x_1,x_2)$, where $R$ equals $=$).  Besides the standard logical operators
for negation ($\neg$), disjunction ($\vee$) and implication ($\Rightarrow$),
this definition also includes modal operators for necessary DRSs ($\Box$),
and possible DRSs ($\Diamond$), as well as a propositional condition
associating a variable ranging over possible worlds with a DRS, which can be
used to represent sentential complements
\citeaffixed{bos2003implementing}{cf.}.


\begin{definition}[Basic DRS] \label{def:bDRS}~\\
A Basic DRS is a tuple $\langle \{x_1 ... x_n\},\{\gamma_1 ... \gamma_m\} 
\rangle$, where:
 \begin{enumerate}[i.]
  \item $\{x_1 ... x_n\}$ is a finite set of variables;
  \item $\{\gamma_1 ... \gamma_m\}$ is a finite set of DRS conditions (which
    may be either basic or complex);
  \item\label{def:bDRS:Rel} $R(x_i, ..., x_j)$ is a basic DRS condition,
    with $x_i ... x_j$ are variables and $R$ is a relation symbol for an
    $n$-place predicate;
  \item $\neg K$, $\Box K$ and $\Diamond K$ are complex DRS conditions, with
    $K$ is a DRS;
  \item $K_1 \vee K_2$ and $K_1 \Rightarrow K_2$ are complex DRS conditions,
    with $K_1$ and $K_2$ are DRSs;
  \item \label{def:bDRS:Prop} $x:K$ is a complex DRS condition, with $x$ is
    a variable and $K$ is a DRS.
 \end{enumerate} 
\end{definition}

\noindent The variables in a DRS are also called \textit{DRS referents} or
\textit{discourse referents}, and the set of referents is called the
\textit{universe} of the DRS.  There are several representations that can be
used to represent a DRS besides its set-theoretical representation. The box
representation shown in examples \ref{ex:donkey}-\Last is the most widely
used representation, but in some cases a linear representation is used in
order to save space. Example~\Next shows the three different
representations for a single sentence:

\ex. John is a vegan.
\a. $\langle \{x, y\},\{$John$(x),$vegan$(y), =(x,y)\}\rangle$
\b. \drs{$x~~y$}{John($x$)\\ vegan($y$)\\ $x=y$}
\c. \flatdrs{$x, y$}{John($x$), vegan($y$), $x=y$}

Note that, for reasons of clarity, we here use a different convention for
representing the attributive use of indefinite noun phrases than in the
PDRS in Example~\ref{ex:binding} above; while the indefinite here introduces
a new discourse referent, which is equated with the variable introduced by
subject, in Example~\ref{ex:binding} this equation is eliminated, resulting
in a representation with only one discourse variable.  These representations
are truth-conditionally equivalent.

Moving to the theory of PDRT, the main difference from DRT is that a basic
PDRS introduces a label and all referents and conditions are associated with
a pointer, which may be bound by the label of an accessible PDRS. Pointers
may also occur free, and they can introduce dependencies between contexts,
which are represented in the set of Minimally Accessible Projection
contexts, or MAPs for short. Thus, in set-theoretic terms, a basic PDRS is
a quadruple that consists of a label, a set of MAPs, a set of projected
referents (i.e., DRS referents associated with a pointer), and a set of
projected conditions. This is formalized as follows:

\begin{definition}[Basic PDRS] \label{def:bPDRS}~\\
A Basic PDRS is a quadruple $\langle l, \{\mu_1 ... \mu_n\}, 
\{\delta_1 ... \delta_m\}, \{\chi_1 ... \chi_k\}\rangle$, where:
  \begin{enumerate}[i.]
    \item $l$ is a projection variable;
    \item $\{\mu_1 ... \mu_n\}$ is a finite set of MAPs, with $\mu_i=\langle
      v_1,v_2\rangle$, and  $v_1$ and $v_2$ are projection variables;
    \item $\{\delta_1 ... \delta_m\}$ is a finite set of projected
      referents, with $\delta_j=\langle v_j, x_j\rangle$, such that $v_j$ is
      a projection variable, and $x_j$ is a DRS referent;
    \item $\{\chi_1 ... \chi_k\}$ is a finite set of projected conditions,
      with $\chi_l = \langle v_l,\gamma_l\rangle$, such that $v_l$ is a
      projection variable, and $\gamma_l$ is a PDRS condition (which may be
      either basic or complex);
    \item \label{def:bPDRS:Rel} $R(x_i, ..., x_j)$ is a basic PDRS condition,
      with $x_i ... x_j$ are variables and $R$ is a relation symbol for an
      $n$-place predicate;
    \item $\neg P$, $\Box P$ and $\Diamond P$ are complex PDRS conditions,
      with $P$ is a PDRS;
    \item $P_1 \vee P_2$ and $P_1 \Rightarrow P_2$ are complex PDRS
      conditions, with $P_1$ and $P_2$ are PDRSs;
    \item\label{def:bPDRS:Prop} $x:P$ is a complex PDRS condition, with $x$
      is a variable and $P$ is a PDRS;
  \end{enumerate}
\end{definition}

\noindent Note the direct correspondence between Definition
\ref{def:bPDRS}(\ref{def:bPDRS:Rel}-\ref{def:bPDRS:Prop}) and Definition
\ref{def:bDRS}(\ref{def:bDRS:Rel}-\ref{def:bDRS:Prop}); the only difference
is that complex PDRS conditions contain subordinated PDRSs, and complex DRS
conditions contain subordinated DRSs.  The definition of projected referents
and projected conditions as tuples allows us to address the two elements
of the tuples separately; we refer to the first element of all projected
referents and projected conditions as the \textit{pointer}, and the second
elements are called the \textit{PDRS referent} and \textit{PDRS condition},
respectively. This is useful, since in some cases we are interested in the
entire projected referent/condition, whereas in other cases we need only to
refer to one of the elements of the tuple. Similarly, we refer to the first
element of the quadruple representing a PDRS, as the \textit{label} of the
PDRS (formally, the label of a PDRS $P_i$ is represented as $lab(P_i)$).
We will refer to labels and pointers together as \textit{projection
variables}.

Just like in DRT, we can represent PDRSs using three different kinds of
representations: a set-theoretic representation, a box representation
and a linear representation, as illustrated in \Next.

\ex. John is a vegan.
\a. $\langle 2, \{\langle 2,1\rangle\}, \{\langle 1, x\rangle, 
      \langle 2, y\rangle\},\{\langle1, $John$(x)\rangle,\langle2, 
      $vegan$(y)\rangle, \langle 2, =(x,y)\rangle\}\rangle$
\b. \pdrs{$2$}{$1\gets x~~2\gets y$}{$1\gets$John($x$)\\ 
      $2\gets$vegan($y$)\\ $2\gets x=y$}{$2\leq 1$}
\c. \flatpdrs{$2$}{$1\gets x, 2\gets y$}{$1\gets$John($x$), 
      $2\gets$vegan($y$), $2\gets x=y$}{$2\leq 1$}


\subsection{Accessibility of Contexts}

One of the central notions in DRT is the notion of accessibility between
DRSs. Accessibility determines which DRS universes are accessible from a DRS
condition in order to bind its referents. This is crucial for determining
which conditions hold for some given referent.  Accessibility between DRSs
is standardly defined based on a subordination relation between DRSs: 

\begin{definition}[DRS Accessibility]~\\
DRS $K_j$ is accessible from DRS $K_i$ ($K_i \leq K_j$) iff
  \begin{itemize}
    \item $K_i = K_j$;
    \item $K_i < K_j$ ($K_i$ is subordinated by $K_j$; 
      see Definition~\ref{def:DRSsub}).
  \end{itemize}
\end{definition}

\noindent Where subordination is recursively defined as follows:

\begin{subdefinition}[DRS Subordination]\label{def:DRSsub}~\\
DRS $K_1$ is subordinated by DRS $K_2$ ($K_1 < K_2$) iff:
  \begin{itemize}
    \item $K_1 <!~K_2$ 
      ($K_1$ is directly subordinated by $K_2$; 
      see Definition~\ref{def:DRSdsub});
    \item There is a DRS $K_3$, such that $K_1 < K_3$  and $K_3 < K_2$.
  \end{itemize}
\end{subdefinition}

\noindent In other words, a DRS is always accessible to itself, and to any
DRS that directly, or indirectly subordinates it.  Direct subordination
between two DRSs $K_1$ and $K_2$ ($K_1 <!~K_2$) means that $K_1$ either
occurs in a condition in $K_2$, or that $K_2$ serves as the antecedent of
$K_1$ in an implication:

\begin{subdefinition}[DRS Direct Subordination]\label{def:DRSdsub}~\\
DRS $K_1$ is directly subordinated by DRS $K_2$ ($K_1 <!~K_2$) iff:
  \begin{itemize}
    \item $\neg K_1$, $\Box K_1$ or $\Diamond K_1$ is a DRS condition of
      $K_2$;
    \item $x:K_1$ is a DRS condition of $K_2$ for some $x$;
    \item $K_1 \Rightarrow K_3$, $K_3 \Rightarrow K_1$, $K_1 \vee K_3$, or 
      $K_3 \vee K_1$ is a DRS condition in $K_2$ for some DRS $K_3$;
    \item There is a DRS $K_3$, such that $K_2 \Rightarrow K_1$
      is a DRS condition in $K_3$.
  \end{itemize}
\end{subdefinition}

For the accessibility relations in PDRT we need to take into account the
contexts introduced by the projection pointers. In the case a pointer occurs
as a free variable, it introduces a \textit{projected context}, which means
that the context is yet to be determined, based on the constraint that it is
accessible from the (current) context of introduction.  Moreover, additional
constraints on accessibility may be determined by the MAPs in the PDRSs from
which the projected context is accessible.
%%XXX mention graph here?
Therefore, we define accessibility over \textit{PDRS-contexts}, which
include those contexts referred to by the label of (accessible) PDRSs, as
well as the projected contexts created by freely occurring pointers.  In
line with DRT accessibility, accessibility between PDRS-contexts is defined
on the basis of subordination, with the addition of the accessibility
relations contributed by the minimally accessible contexts (MAPs):

\begin{definition}[PDRS-context Accessibility]~\\
PDRS-context $\pi_j$ is accessible from PDRS-context $\pi_i$ 
($\pi_i \leq \pi_j$) iff
  \begin{itemize}
    \item $\pi_i = \pi_j$;
    \item $\pi_i < \pi_j$ ($\pi_i$ is subordinated by $\pi_j$; see
      Definition~\ref{def:PDRSsub});
    \item $\langle \pi_1,\pi_2 \rangle$ is an element of the set of
     MAPs of some PDRS $P_k$;
  \end{itemize}
\end{definition}

\noindent Subordination is again defined recursively, directly corresponding
to Definition~\ref{def:DRSsub}.

\begin{subdefinition}[PDRS-context Subordination]\label{def:PDRSsub}~\\
PDRS-context $\pi_1$ is subordinated by PDRS-context $\pi_2$
($\pi_1 < \pi_2$) iff:
\begin{itemize}
   \item $\pi_1 <!~\pi_2$ ($\pi_1$ is directly subordinated by $\pi_2$; 
     see Definition~\ref{def:PDRSdsub});
   %\item $\pi_1 \prec \pi_2$ ($\pi_2$ is a minimally accessible context
   %  with respect to $\pi_1$; see Definition~\ref{def:PDRSmac});
   \item There is a PDRS-context $\pi_3$, such that $\pi_1 < \pi_3$ and
     $\pi_3 < \pi_2$.
\end{itemize}
\end{subdefinition}

\noindent In PDRT, direct subordination is not determined by the context in
which a condition containing a sub-PDRS occurs (as in the case of DRT; see
Definition~\ref{def:DRSdsub}), but by the context in which the condition is
interpreted, i.e., the pointer associated with the condition. Therefore,
direct subordination for PDRS-contexts is defined over projected conditions:

\begin{subdefinition}[PDRS-context Direct Subordination]\label{def:PDRSdsub}~\\
PDRS-context $\pi_1$ is directly subordinated by PDRS-context $\pi_2$ 
($\pi_1 <!~\pi_2$) iff:
  \begin{itemize}
    \item $\langle\pi_2,\neg P_i\rangle$,
      $\langle\pi_2,\Box P_i\rangle$,
      or $\langle\pi_2,\Diamond P_i\rangle$ is a projected condition in
      some PDRS $P_k$, s.t. $lab(P_i) = \pi_1$;
    \item $\langle\pi_2,x:P_i\rangle$ is a projected condition in some
      PDRS $P_k$, for some $x$, such that $lab(P_i) = \pi_1$;
    \item  $\langle\pi_2,P_i \Rightarrow P_j\rangle$ or
      $\langle\pi_2,P_i \vee P_j\rangle$ is a projected condition in
      some PDRS $P_k$, such that $lab(P_i) = \pi_1$ or $lab(P_j) = \pi_1$;
    \item $\langle \pi_3, P_i \Rightarrow P_j \rangle$ is a projected condition
      in some PDRS $P_k$, for some $\pi_3$, such that $lab(P_i) = \pi_2$ and 
      $lab(P_j) = \pi_1$.
  \end{itemize}
\end{subdefinition}

%\begin{subdefinition}[Minimally Accessible PDRS-Contexts]~\label{def:PDRSmac}\\
%PDRS-context $\pi_i$ is a minimally accessible PDRS-context with
%respect to $\pi_j$ ($\pi_j \prec \pi_i$) iff:
%  \begin{itemize}
%    \item There is some PDRS $P_k$, such that $\pi_i$ is introduced in $P_k$
%      and $lab(P_k)=\pi_j$;
%    \item $\langle \pi_j,\pi_i\rangle$ is an element of the set of
%      \MAPs~of some PDRS $P_k$;
%      %Moet dit nog gekwalificeerd worden? Ziets als:
%      %, such that $lab(P_k) \leq \pi_j$, and either
%      %$lab(P_k) \leq \pi_i$, or $\pi_i$ occurs free. % check this def!
%  \end{itemize}
%\end{subdefinition}

The accessibility relation ($\leq$) creates a partial order over all
---projected and non-projected--- PDRS-contexts. This results in a directed
graph, called the \textit{Accessibility Graph}, over all PDRS-contexts in
a given PDRS. We will call a PDRS whose accessibility graph is weakly
connected, a \emph{connected} PDRS:

\begin{definition}[Connectedness]~\\
  A PDRS $P$ is connected iff:
  \begin{itemize}
    \item The projection graph $G_P$ of $P$ is weakly connected, i.e., for
      all pairs of vertices ($v_1$,$v_2$) in $G_P$ it holds that if all
      directed edges in $G_P$ are replaced by undirected edges, then there is
      a path between $v_1$ and $v_2$.  
  \end{itemize}
\end{definition}

\noindent All basic PDRSs are connected PDRSs, because projection variables
either occur free (in which case they are accessible from their introduction
site), or are bound by some sub-PDRS, which is necessarily accessible from
the global PDRS. We will see that unresolved structures, introduced below,
may result in accessibility graphs that consist of multiple non-connected
components.
%XXX unresolved Merges may result in a non-connected graph (but then watch
%out for accidental bindings!)
%%XXX Define properly in Sandbox 

%resolved PDRS (no projected contexts): accessibility graph becomes tree.

\subsection{Variable Binding}

Up to this point we have been talking informally about free and bound
referents and projection variables. DRT and PDRT are dynamic semantic
formalisms, which means that the meaning structures simultaneously provide
semantic content and serve as a context for novel information. That is, the
referents introduced in a (P)DRS may serve as antecedents for later
introduced anaphoric expressions. In DRT, a referent is bound in case it is
introduced in the universe of some accessible DRS, as formally defined below.

\begin{definition}[DRS Variable Binding]~\\
DRS variable $x$, introduced in DRS $K_i$, is bound in global DRS $K$ iff:
\begin{quote}
There exists a DRS $K_j \leq K$, such that
\begin{enumerate}[i.]
  \item $K_i \leq K_j$;
  \item $x\in U(K_j)$, where $U(K_j)$ refers to the universe of DRS $K_j$.
\end{enumerate}
\end{quote}
\end{definition}

\noindent All DRS variables that do not meet this requirement are free
variables. In the DRT literature it is common to provide a function over
DRSs that collects all free variables systematically from the available
variables in a DRS
\citeaffixed{kamp1993discourse,bos2003implementing,kamp2011discourse}{see,
e.g.,}. The idea is that the set of free variables of a DRS can be described
as the set of all variables occurring in the body of a DRS (in a DRS
predicate or a propositional condition) minus the variables that occur in an
---accessible--- DRS universe. Since accessibility in DRT is defined
hierarchically, it suffices to collect all variables from the conditions of
a DRS in a bottum-up fashion, and accordingly subtract the variables
occurring in the universe of the DRS.

\begin{definition}[Free DRS Variables]~\label{def:fdrsvs} 
  \begin{enumerate}[i.]
    \item $\mathcal{F}(\langle U, C \rangle)
          = (\bigcup_{c\in C} \mathcal{F}(c)) - U$;
    \item $\mathcal{F}(R(x_1,...,x_n)) = \{x_1,...,x_n\}$;
    \item $\mathcal{F}(\neg K) 
          = \mathcal{F}(\Diamond K) 
          = \mathcal{F}(\Box K)
          = \mathcal{F}(K)$;
    \item $\mathcal{F}(K_1 \Rightarrow K_2)
          = \mathcal{F}(K_1) \cup (\mathcal{F}(K_2) - U(K_1))$;
    \item $\mathcal{F}(K_1 \vee K_2)
          = \mathcal{F}(K_1) \cup \mathcal{F}(K_2)$;
    \item $\mathcal{F}(x:K) = \{x\} \cup \mathcal{F}(K)$.
  \end{enumerate}
\end{definition}

\noindent Note that the different accessibility constraints for implication
and disjunction here become explicit, since the variables in the universe of
the antecedent of an implication can bind the free variables in its
consequent, while this does not hold for the two disjuncts of a disjunction.

In PDRT, we need to define binding over both projection variables and
projected referents. The binding of projection variables can be defined in
the same way as the binding of referents in DRT, with the labels of
accessible PDRSs serving as possible antecedents:

\begin{definition}[Projection Variable Binding]~\\
Projection variable $v$, introduced in PDRS $P_i$, is bound iff:
\begin{quote}
There exists a PDRS $P_j$, such that
\begin{enumerate}[i.]
  \item $lab(P_i) \leq lab(P_j)$; 
  \item $v = lab(P_j)$.
\end{enumerate}
\end{quote}
\end{definition}

\noindent Following Definition~\ref{def:fdrsvs}, we can define a function
for collecting free projection variables. Here, the set of free projection
variables is defined as the set of all projection variables available in the
set of MAPs, the set of projected referents and the set of projected
conditions, minus those occurring as a label in an accessible PDRS.

\begin{definition}[Free Projection Variables]\label{def:def:fpvs}~
  \begin{enumerate}[i.]
    \item $\mathcal{FP}(\langle l, M, U, C \rangle) 
          = (\bigcup_{m\in M} \mathcal{FP}(m)
            \cup \bigcup_{u\in U} \mathcal{FP}(u) 
            \cup \bigcup_{c\in C} \mathcal{FP}(c)) - \{l\}$
    \item $\mathcal{FP}(\langle p_1, p_2\rangle) = \{p_1, p_2\}$
    \item $\mathcal{FP}(\langle p, r\rangle) = \{p\}$
    \item $\mathcal{FP}(\langle p, R(...)\rangle) = \{p\}$
    \item $\mathcal{FP}(\langle p,\neg K\rangle)
          = \mathcal{FP}(\langle p,\Diamond K\rangle)
          = \mathcal{FP}(\langle p,\Box K\rangle)
          = \mathcal{FP}(\langle p,x:K\rangle) 
          = \{p\} \cup \mathcal{FP}(K)$
    \item $\mathcal{FP}(\langle p,K_1 \Rightarrow K_2\rangle) 
          = \{p\}\cup \mathcal{FP}(K_1)
            \cup (\mathcal{FP}(K_2) - lab(K_1))$
    \item $\mathcal{FP}(\langle p,K_1 \vee K_2\rangle) 
          = \{p\}\cup \mathcal{FP}(K_1) \cup \mathcal{FP}(K_2)$
  \end{enumerate}
\end{definition}

For the binding of referents in PDRT, we again need to define whether the
referent occurs in an accessible universe.  However, as illustrated in
example~\ref{ex:binding}, here repeated as \Next, the binding of referents
in PDRT is non-hierarchical in the sense that referents may be bound by
antecedents introduced in a PDRS that is not accessible.

\ex. It is not the case that John is a vegetarian, he's a vegan.\\
\label{ex:binding_rep}
\hspace*{-0.3cm}{
  \pdrs{1}{}{
    $1\gets\neg$\pdrs{2}{$3\gets x$}{
      $3\gets$John($x$)\\ $2\gets$vegetarian($x$)
        }{$2 \leq 3$}\\
      $1\gets$vegan($x$)
  }{$1\leq 3$}
}

In this example, the MAPs describing the accessible PDRS-contexts determine
that the variable $x$ in the predicate \textit{vegan} is bound by the
projected referent introduced in the universe of the PDRS labelled $2$,
despite the fact that $2$ is not an accessible context from PDRS $1$.  So,
since accessibility in PDRT is defined over PDRS-contexts indicated via
projection variables, the binding of referents should also be defined based
on PDRS-contexts. This means that we need to define binding over
\textit{projected referents}, consisting of a pointer and a referent (in the
example above, we thus need to determine whether the projected referent
$\langle 1, x\rangle$ is bound). The pointer determines the interpretation
site of the projected referent, so the accessibility requirement for the
antecedent of a projected referent should hold between the pointer of the
referent and the pointer of its antecedent.  More specifically, a projected
referent $\langle p, r \rangle$ is bound in case there exists a projected
referent $\langle p',r'\rangle$ in some universe, such that $r'=r$ and $p'$
is accessible from $p$.  This is formalized in the following definition.
Here, the set \textit{PVars(P)} represents the set of all projection
variables (labels and pointers) that occur in PDRS $P$.

\begin{definition}[Projected Referent Binding]~\\
Projected referent $\langle p,r\rangle$, introduced in PDRS $P_i$, is bound
in global PDRS $P$ iff:
\begin{quote}
There exists a PDRS-context $\pi_j \in PVars(P)$, such that
\begin{enumerate}[i.]
  \item $p \leq \pi_j$; 
%  \item $lab(P_i) \leq \pi_j$; 
    %XXX is this by definition true, since $lab(P_i) \leq p$ (given a
    %connected PDRS)?
  \item There exists some PDRS $P_j \leq P$, such that 
    $\langle \pi_j,r\rangle \in U(P_j)$.
\end{enumerate}
\end{quote}
\end{definition}

\noindent In other words, a projected referent is bound in case there exists
some referent pointing to an accessible context in any universe in the
global PDRS.  Since by definition the interpretation context of some
semantic content is accessible from its context of introduction, it follows
from this definition that the projection context of the antecedent is also
accessible from the introduction site of the projected referent ($lab(P_i)
\leq \pi_j$). 

In order to collect the free projected referents of a PDRS in the same
systematic way as the Definitions~\ref{def:fdrsvs} and \ref{def:fpvs}
describe for DRS referents and PDRS projection variables, respectively, we
need to find a way to account for the non-hierarchical binding illustrated
in example~\Last.  Therefore, we define the set of \textit{free projected
referents} of a PDRS $P$ is defined as the difference between the set of
\textit{all projected referents} ($\mathcal{A}$) of $P$ and the set of
\textit{bound projected referent candidates} ($\mathcal{B}$) of $P$, which
are the projected referents that occur in some universe in $P$, plus those
created by associating each referent with the projection variables of all
contexts accessible from its interpretation site.
%%XXX probably need to explain this better

\begin{definition}[PDRS Free Projected Referents]~
  \begin{itemize}
    \item $\mathcal{FR}(P) = \mathcal{A}(P) - \mathcal{B}(P)$.
  \end{itemize}
\end{definition}

\noindent The collection of all projected referents of a PDRS is
straightforward: for each (embedded) PDRS, we take the union of the universe
of that PDRS and the referents that occur in a predicate or a propositional
condition together with the pointer of their respective conditions. This is
formally described as follows:

\begin{subdefinition}[All Projected Referents]~
  \begin{enumerate}[i.]
    \item $\mathcal{A}(\langle l, M, U, C \rangle)
      = U \cup (\bigcup_{c\in C} \mathcal{A}(c))$
    \item $\mathcal{A}(\langle p, R(x_1,...,x_n)\rangle)
      = \{\langle p, x_1\rangle, ..., \langle p, x_n\rangle\}$
    \item $\mathcal{A}(\langle p,\neg K\rangle) 
      = \mathcal{A}(\langle p,\Diamond K\rangle) 
      = \mathcal{A}(\langle p,\Box K\rangle)
      = \mathcal{A}(K)$
    \item $\mathcal{A}(\langle p,K_1 \Rightarrow K_2\rangle)
      = \mathcal{A}(\langle p,K_1 \vee K_2\rangle)
      = \mathcal{A}(K_1) \cup \mathcal{A}(K_2)$
    \item $\mathcal{A}(\langle p,x:K\rangle)
      = \{\langle p,x \rangle\} \cup \mathcal{A}(K)$
  \end{enumerate}
\end{subdefinition}

As decribed above, in order to determine the set of \textit{bound projected
referent candidates} of PDRS $P$, we collect not only the projected
referents actually occurring in the universe of $P$, but also the
combinations of the referents with all contexts from which their projection
context is accessible. Since the availability of accessible contexts is not
hierarchical (i.e., the projection context that binds a referent may be
introduced in some embedded PDRS), we need to keep track of the entire
accessibility graph in order to create all possible bindings; we can do this
by keeping track of the MAPs of all superordinate PDRSs and create possible
bindings based on these accessibilities. The collection and generation of
bound projected referent candidates is shown below. 

\begin{subdefinition}[Bound Projected Referent Candidates]\label{def:bprcs}~\\
  The possibly bound projected referents are determined on the basis of a
  set of projected referents ($R$), and a set of MAPs representing
  accessibility between contexts ($A$).
  \begin{enumerate}[i.]
    \item $\mathcal{B}(\langle l, M, U, C \rangle) 
      = \mathcal{B'}(\langle l, M, U, C \rangle,\emptyset,\emptyset)$
    %\item $\mathcal{B'}(\langle l, M, U, C \rangle,R,A) 
    %  = R' \cup (\bigcup_{\langle p, r\rangle\in R'} \langle l, r\rangle)
    %  \cup (\bigcup_{\langle p_1,p_2\rangle\in A'} R'[p_2\backslash p_1])
    %  \cup (\bigcup_{c\in Acc(C)} \mathcal{B'}(c,R',A'))
    %  \cup (\bigcup_{c\notin Acc(C)} \mathcal{B'}(c,\emptyset,\emptyset))$\\
    %  where: $R' = R \cup U$;
    %         $A' = A \cup M$;
    %         $Acc(C) = \{\langle p, c'\rangle \in C~|~ p \leq l\}$
    \item $\mathcal{B'}(\langle l, M, U, C \rangle,R,A) = 
          R'
          \cup \mathcal{B'}(l,R',A)
          \cup \mathcal{B'}(M,R',A)
          \cup \mathcal{B'}(C,R',A\cup M)$\\
      where: $R' = R\cup U$
    \item $\mathcal{B'}(l,R,A) 
      = \bigcup_{\langle \_, r\rangle\in R} \langle l, r\rangle$
    \item $\mathcal{B'}(M,R,A)
      = \bigcup_{\langle p_1,p_2\rangle\in M\cup A} R[p_2\backslash p_1]$
    \item $\mathcal{B'}(C,R,A)
      = (\bigcup_{c\in Acc(C)} \mathcal{B'}(c,R,A))
      \cup (\bigcup_{c\notin Acc(C)} \mathcal{B'}(c,\emptyset,\emptyset))$\\
      where: $Acc(C) = \{\langle p, \_\rangle \in C~|~ p \leq l\}$
    \item $\mathcal{B'}(\langle p, R(...)\rangle,R,A) = \emptyset$
    \item $\mathcal{B'}(\langle p,\neg K\rangle,R,A)
      = \mathcal{B'}(\langle p,\Diamond K\rangle,R,A)
      = \mathcal{B'}(\langle p,\Box K\rangle,R,A)
      = \mathcal{B'}(\langle p,x:K\rangle,R,A)
      = \mathcal{B'}(K,R,A)$
    \item $\mathcal{B'}(\langle p,K_1 \Rightarrow K_2\rangle,R,A)
      = \mathcal{B'}(K_2,\mathcal{B'}(K_1,R,A),A\cup \MAPs(K_1))$
    \item $\mathcal{B'}(\langle p,K_1 \vee K_2\rangle,R,A)
      = \mathcal{B'}(K_1,R,A) \cup \mathcal{B'}(K_2,R,A)$
      \end{enumerate}
\end{subdefinition}

\noindent Informally, the algorithm described in Definition~\ref{def:bprcs}
works as follows: all previously found referents, plus the universe of the
current PDRS are collected, and these are enhanced with hypothetical
antecedents created on the basis of the current label, and the current set
of MAPs (by adding new projected referents for those referents whose pointer
occurs as the second element of a MAP). In addition, all universes and
hypothetical referents of the PDRSs embedded in the conditions are
collected, based on the referents in the universes and the MAPs obtained so
far. Note that projected conditions (i.e., those conditions $c$ for which it
holds that $c \notin Acc(C)$) are not accessible to the bound referents
obtained so far, and therefore do not take these into account for creating
possible bound projection referents.

\subsection{Properties of DRSs and PDRSs}

Based on the definitions of free and bound variables in DRT and PDRT, we can
define several properties of DRSs and PDRSs. Firstly, a (P)DRS without any
free discourse referents is called a \textit{proper} (P)DRS. In general,
a (P)DRS representing a complete sentence or preposition will be proper,
meaning that none of the referents remain underspecified. As we will see in
Section~\ref{sec:unresolved}, unresolved structures representing lexical
items may contain (P)DRSs that are not (yet) proper, and thus contain free
referents.

\begin{definition}[Properness]~\\
A DRS $K$/PDRS $P$ is proper iff:
\begin{itemize}
  \item $K$/$P$ does not contain any free referents: 
    $\mathcal{F}(K) = \emptyset$ / $\mathcal{FR}(P) = \emptyset$.
\end{itemize}
\end{definition}

In PDRT, we have defined free and bound referents, as well as free and bound
projection variables. This provides us with an extra level of information
about the projective properties of PDRSs. In particular, a PDRS without any
free projection variables does not contain any unresolved presuppositions,
which we will call a \textit{non-presuppositional} or \textit{simple} PDRS.

\begin{definition}[Simpleness]~\\
A PDRS $P$ is simple iff
\begin{itemize}
  \item $P$ does not contain any free pointers: $\mathcal{FP}(P) = \emptyset$.
\end{itemize}
\end{definition}

\noindent Note that in a simple PDRS not all content needs to be asserted;
some pointers may be bound by labels of accessible contexts. Thus, we can
also define a property that describes PDRSs with only asserted content, we
will call these PDRSs \textit{projectionless}, or \textit{plain}.
%or: settled

\begin{definition}[Plainness]
A PDRS $P$ is plain iff
\begin{itemize}
  \item All projection variables in $P$ are locally accommodated, i.e.,
    $\forall P' \leq P: \mathcal{FP}(P') = \emptyset$.
\end{itemize}
\end{definition}

\noindent Following these definitions, all plain PDRSs are simple PDRSs, but
not all simple PDRSs are plain. Simple, non-plain PDRSs are those PDRSs in
which some content is accommodated at a higher level, but not presupposed
(also referred to as \textit{intermediate accommodation}).
%XXX illustrate?

Finally, the property of \emph{purity} refers to the occurrence of duplicate
uses of variables. In DRT, if a variable occurs as DRS referent in multiple
accessible universes, this may result in ambiguous bindings; the same holds
for duplicate usages of projection variables as labels in PDRT.

\begin{definition}[Purity]~\\
A DRS $K$/PDRS $P$ is pure iff:
\begin{itemize}
  \item $K$/$P$ does not contain any otiose declarations of (projection)
    variables (i.e., $K$/$P$ does not contain any unbound, duplicate uses
    of variables).
    \begin{itemize}
      \item For all $K_1$, $K_2$, such that $K_1 < K_2 \leq K$ it holds that:
        $U(K_1) \cap (U(K_2) \cup \mathcal{F}(K)) = \emptyset$
      \item For all $P_1$, $P_2$, such that $P_1 < P_2 \leq P$ it holds that: 
        $\{lab(P_1)\} \cap (\{lab(P_2)\} \cup \mathcal{FP}(P)) = \emptyset$
    \end{itemize}
\end{itemize}
\end{definition}

%%%XXX something about why duplicate uses of (projected) referents in PDRT is
%not a problem.
