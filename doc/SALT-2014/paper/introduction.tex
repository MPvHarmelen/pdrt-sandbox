\section{Introduction}\label{Introduction}

In the literature on projection, presuppositions have received the most
attention in terms of describing and explaining their projection behaviour.
However, since Potts' redefinition of the class of conventional implicatures
\citep{potts2003logic,potts2005logic}, the interest has shifted toward the
broader class of phenomena that exhibit the property of projection.

%XXX 
\bigskip
\begin{tabular}{p{0.2\textwidth} p{0.7\textwidth}}
          \textbf{Goal:} & A unified account of projection phenomena.\\
     \textbf{Challenge:} & CIs provide \textit{new} information, but they
                           project as if providing \textit{old} information
                           (like presuppositions).\\
   \textbf{Observation:} & Supplemental CIs occur with a projecting anchor,
                           and always project as far as their anchor.\\
      \textbf{Solution:} & CIs provide \textit{new} information to the
                           (projecting) context created by a presupposition,
                           namely the one provided by its anchor.\\
\textbf{Implementation:} & CIs are projection-anaphoric in the sense that
                           they inherit their projection context from their
                           (projecting) anchor. This is formalized in PDRT
                           \citep{venhuizen2013iwcs} by introducing a
                           dependency at the lexical level between the
                           projection site of the CI and the projection site
                           of the anchor.\\
     \textbf{Next Step:} & Other CIs: expressives and supplemental adverbs.\\
   \textbf{Solution II:} & Subjective PDRT with pointers restricted to speaker
                           models.\\
    \textbf{Discussion:} & Are CIs a homogeneous class? What is their relation
                           to presuppositions and anaphora?\\
\end{tabular}\\

References: \citep{delgobbo2003appositives,potts2005logic,amaral2007review,
  nouwen2007appositives,harris2009perspective,heringa2012appositional,
  schlenker2013supplements,nouwen2014note,potts2013presupposition}





