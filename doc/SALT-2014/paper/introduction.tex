\section{Introduction}\label{Introduction}

In the literature on projection, presuppositions have received the most
attention in terms of describing and explaining their projection behaviour.
However, since Potts' redefinition of the class of conventional implicatures
\citep{potts2003logic,potts2005logic}, the interest has shifted toward the
broader class of phenomena that exhibit the property of projection.

%XXX 
\bigskip
\begin{tabular}{p{0.2\textwidth} p{0.7\textwidth}}
  \textbf{Challenge:} & CIs provide \textit{new} information, but they
                        project as if providing \textit{old} information
                        (like presuppositions).\\

   \textbf{Solution:} & CIs provide \textit{new} information to the
                        (projecting) context created by a presupposition,
                        namely the one provided by its anchor.\\

\textbf{Implementation:} & CIs are projection-anaphoric in the sense that
                        they inherit their projection context from their
                        (projecting) anchor. This is formalized in PDRT
                        \citep{venhuizen2013iwcs} by introducing a dependency
                        at the lexical level between the projection site of
                        the CI and the projection site of the anchor.
\end{tabular}

\subsection{\ldots}

\subsection{Supplemental CIs in discourse}

\begin{itemize}
  \item CIs always occur with an anchor
  \item Subordinated constructions
  \item Types of anchors (generic vs. (in)specific indefinite vs definite
    vs. \ldots)
  \item CIs tend to project globally (scopelessness)
  \item CIs are non-cancellable
  \item CIs are non-restrictive
  \item Speaker-orientedness
  \item Similarity to E-type anaphora
  \item CIs resist binding
\end{itemize}

\citep{delgobbo2003appositives,nouwen2007appositives,
heringa2012appositional,schlenker2013supplements}


\citep{potts2005logic,amaral2007review,harris2009perspective,nouwen2014note}

