\documentclass{salt}

% This document template was developed by the editors of /Semantics and
%  Pragmatics/ and minimally modified to work with the salt.cls document
%  class.

%=====================================================================
%============================= packages ==============================

\usepackage{natbib}
\usepackage{linguex}
\renewcommand\firstrefdash{}

%=====================================================================
%============================== macros ===============================

\newcommand{\drs}[2]
{\footnotesize\begin{tabular}{c}
  \noalign{\smallskip}
  \begin{tabular}{|l|}
    \hline #1\\ \hline #2\\ \hline
  \end{tabular}
  \smallskip
\end{tabular}
\normalsize}

\newcommand{\pdrs}[4]
{\footnotesize\begin{tabular}{c}
  \noalign{\smallskip}
  $#1$\\
  \begin{tabular}{|l|}
    \hline #2\\ \hline #3\\ \hline #4 \\ \hline
  \end{tabular}
  \smallskip
\end{tabular}
\normalsize}

%=====================================================================
%========================= preamble material =========================

% Metadata for the PDF output. ASCII-only!
\pdfauthor{Noortje J. Venhuizen, Johan Bos, Petra Hendriks, Harm Brouwer}
\pdftitle{How and why conventional implicatures project}
\pdfkeywords{conventional implicatures, CIs, DRT, PDRT, projection}

% Optional short title inside square brackets, for the running headers.
% If no short title is given, no title appears in the headers.
\title[How and why CIs project]{% in sentence case (down style)
  How and why conventional implicatures project 
  \thanks{We thank \ldots}}

% Optional short author (last name only) inside square brackets, for the running headers.
% If no short author is given, no authors print in the headers.
\author[Venhuizen et al.]{% As many authors as you like, each separated by \AND.
  \saltauthor{Noortje J. Venhuizen \\ \institute{University of Groningen}} \AND
  \saltauthor{Johan Bos \\ \institute{University of Groningen}} \AND
  \saltauthor{Petra Hendriks \\ \institute{University of Groningen}} \AND
  \saltauthor{Harm Brouwer \\ \institute{University of Groningen}}%
}

%=====================================================================

\begin{document}

%=====================================================================
%============================ frontmatter ============================

\maketitle

%--------------------------------------------------------------------
% First page headers and page numbers
%
% the page number of the first page of this paper
% \setcounter{page}{1}

% Create the first page headings.
% This needs to be issued *after* \maketitle.
%       {volume}{first page}{last page}{year}{not used}{not used}
\firstpageheadings{24}{000}{000}{2014}{}{}
%
%
%---------------------------------------------------------------------

\begin{abstract}  
  Abstract here
\end{abstract}

\begin{keywords}
  conventional implicatures, CIs, DRT, PDRT, projection %(special formatting is fine)
\end{keywords}

\tableofcontents %%XXX comment out in final version

%=====================================================================
%============================ article text ===========================

\section{Introduction}

Semantic frameworks aim to be wide coverage, as well as formally adequate.
DRT: one of the most influential.

Formally well-defined: model-theoretic backbone
\cite{kamp1981theory,kamp1993discourse}, compositional construction
\cite{muskens1996combining}.

Wide coverage: anaphora, tense \cite{kamp1981theory}, quantification \&
plurality \cite{kamp1993discourse}, attitude reports
\cite{asher1986belief,asher1989belief,zeevat1996neoclassical,maier2009presupposing}
discourse structure \cite{asher2003logics}, and presupposition
\cite{sandt1992presupposition,kamp1994drs,krahmer1998presupposition,geurts1999presuppositions}

We have proposed a novel treatment of presupposition in which it is treated
as part of a larger class of projection phenomena, does not require
a two-stage resolution strategy, and distinction between asserted and
projected content (i.e., explicit information structure): PDRT
\cite{venhuizen2013iwcs}.  Pointers indicate interpretation site. Can be
shown to account for various projection phenomena.  However, adding
projection pointers affects fundamental aspects of DRT non-trivially:
accessibility, binding, construction. 

In this paper, we will describe the formal implications of extending DRT
with projection pointers. After briefly describing the motivation behind
this DRT extension in Section~\ref{sec:background}, we systematically
derive the formal definitions of PDRT from the DRT definitions for building
and combining structures, and show where they diverge
(Sections~\ref{sec:building} and~\ref{sec:combining}).  By providing
a translation from PDRT to DRT in Section~\ref{sec:translating}, we show
that PDRT inherits all formal interpretive properties of DRT, together with
its inference mechanisms.  We introduce an implementation of DRT and PDRT,
called \textsc{pdrt-sandbox}, which implements all these definitions, and
provides a full-fledged toolkit for use in NLP applications. In
Section~\ref{sec:playing}, we offer some directions for applying the
formalism and implementation. Finally, Section~\ref{sec:conclusions}
concludes the paper.






%Presuppositions are a fundamental aspect of linguistic meaning since they
%relate the content of an utterance to the unfolding discourse. A definite
%description like ``the man'', for example, introduces a male discourse
%entity that is salient in the current discourse context.  Yet, in many
%formal semantic accounts, these phenomena have been treated as deviations
%from standard meaning construction, e.g., by resolving them only in
%a second step of processing \cite{sandt1992presupposition}. This is
%unsatisfactory for a number of reasons, mainly because it precludes an
%incremental compositional treatment of projection phenomena such as
%presuppositions.  Therefore, we have recently proposed Projective DRT
%(PDRT), which builds upon the widely used Discourse Representation Theory
%\cite{kamp1981theory,kamp1993discourse}.  PDRT deals with projected content
%during discourse construction, without losing any information about where
%information is introduced or interpreted \cite{venhuizen2013iwcs}.

%In order to evaluate its properties and applications, we implemented PDRT as
%a Haskell library, called \textsc{pdrt-sandbox}. This library includes machinery
%for representing PDRSs, as well as traditional DRSs, a translation from PDRT
%to DRT and FOL, and the different types of merge operations for (P)DRSs.
%Unresolved structures with lambda-abstractions \cite{muskens1996combining} are
%implemented as pure Haskell functions by exploiting Haskell's
%lambda-theoretic foundations. Several (P)DRS representations can be chosen
%as target output, including the well-known box-representations,
%a set-theoretic notation and flat, non-recursive, table-structures.
%\textsc{pdrt-sandbox} provides a comprehensive toolkit for use in NLP applications,
%and paves way for a more thorough understanding of the behaviour of
%projection phenomena in language.  


\section{Explaining the projection behaviour of CIs}

\subsection{CIs, presuppositions and anaphora}

\citet{sandt1992presupposition}

\citep{potts2013presupposition}

\subsection{Projection via anchoring}

CIs are projection-anaphoric to their anchor.




\section{Beyond supplements}

\subsection{Speaker-orientedness in PDRT}

Following the traditional model-theoretic interpretation of DRT \citep[see,
e.g.]{kamp1981theory,kamp1993discourse,kamp2011discourse}, a Discourse
Representation Structure (DRS) is thought of as a \textit{partial model}
representing the information conveyed by some utterance, which needs to be
embedded in a \textit{total model} to determine its interpretation. This
embedding is done via an \textit{embedding function} from the set of
discourse referents in the universe of the DRS $K$ into elements in the
domain of the total model $M$, in such a way that all conditions of $K$ are
verified in $M$ (with respect to some world $w$).

To implement speaker-orientation, we can assume that each speaker $S$ has
its own model $M_S$ describing the current state of affairs of the world.
These models may or may not coincide with the model representing the
\textit{actual} state of affairs, which is represented by the `Truth Model'
$M_T$. The idea is that the interpretation a PDRS condition may be
restricted to some (subjective) model, meaning that it only needs to hold in
the designated model. Since in PDRT the information about the interpretation
of linguistic content is explicitly available (in the form of
\textit{pointers}), an interpretational restriction with respect to some
model may also be incorporated as part of the pointer, e.g. via a subscript.
This thus means that for some condition with a pointer of the form `$p_S$',
it holds that this condition is only required to hold with respect to the
model of speaker $S$. Thus, when determining the truth of some utterance
containing some subjective content (i.e., with respect to the Truth Model
$M_T$), the subjective content has to be fulfilled only with respect to the
subjective model $M_S$; in case this holds, the entire utterance will come
out true (provided, of course, that all other conditions are verified with
respect to $M_T$). Note that this places a constraint on the domains of
$M_S$ and $M_T$ insofar that they must overlap with respect to the
referent(s) mentioned in the subjective condition.

\subsection{Other types of CIs}
\subsubsection{Expressives}

We focus on supplemental CIs since these are the most common and
well-studied type of CI.  However, a case can be made for a similar account
of expressives \citep[including expressive attributive adjectives, epithets,
honorifics, and tense-variations such as the German `Konjunktiv I';
cf.][]{potts2005logic}. The status of expressives has been open to some
debate \citep[...]{potts2004japanese,geurts2007fucking}.  But like
supplemental CIs, expressives in general require some antecedent (or:
anchor) to express some opinion about.
%%examples (see, e.g. Amaral et al. p. 711)
However,
the notion of speaker-orientedness is especially important in this context,
so a framework is required that can incorporate speaker attitudes.
[...]
%Q: Do expressives require a projecting anchor?

\noindent\parbox[b]{\textwidth}{\paragraph{Expressive attributive adjectives}
\ex. Sue's dog is \underline{fucking} mean.\\
a.~~~\pdrs{1}{$2\gets$x~~$3\gets$y}{$2\gets$Sue(x)\\ $3\gets$dog(y)\\
  $3\gets$of(y,x)\\ $1\gets$mean(y)\\ $4_S\gets$fucking\_mean(y)}{
  $1<2$~~$1<3$~~$1<4$~~$4=3$}
~~~b.~~~\pdrs{1}{$2\gets$x~~$3\gets$y~~$1\gets$e}{$2\gets$Sue(x)\\ $3\gets$dog(y)\\
  $3\gets$of(y,x)\\ $1\gets$mean(e)\\ $1\gets$Topic(e,y)\\
  $1_S\gets$fucking(e)}{$1<2$~~$1<3$}

}

\noindent In PDRS \Last[a], the anchor of the CI is ``Sue's dog'', which
projects.  However, the representation is not completely satisfactory, since
the adjective ``mean'' is duplicated: it occurs once as part of the asserted
content, and once as part of the expressive CI content. Moreover, the
contraction of ``fucking'' and ``mean'' serving as one predicate goes
against the compositional idea in (P)DRT. In \Last[b], this tension is
resolved by using a neo-Davidsonian event-structure for the adjectival
construction; by letting the adjective ``mean'' introduce an event, which is
related to the subject using VerbNet roles \citep{kipper2008large}, the
expressive adjective ``fucking'' can manipulate the event itself, instead of
the subject. This provides a more intuitive distinction between the asserted
content and the expressive content of this utterance. However, the direct
correspondence to supplemental CIs, which require a projecting anchor, seems
to be obfuscated with this representation.

\noindent\parbox[b]{\textwidth}{\paragraph{Epithets}
\ex. Every democrat advocating a proposal for reform says 
  \underline{the stupid thing} is worthwhile.\\
a.~~\pdrs{1}{}{$1\gets$\pdrs{2}{$2\gets$x~~$2\gets$y}{$2\gets$Democrat(x)\\ 
  $2\gets$proposal\_for\_reform(y)\\ $2\gets$advocate(x,y)}{}$\Rightarrow
  $\pdrs{3}{$3\gets$p}{$3\gets$says(x,p)\\ 
  $3\gets$p:\pdrs{4}{$4,5_S\gets$z}{$5_S\gets$stupid\_thing(z)\\ $4,5_S\gets$z=y\\
  $4\gets$worthwhile(z)}{$4<5$~~$5=2$}}{$3<2$}}{}
b.~~\pdrs{1}{}{$1\gets$\pdrs{2}{$2\gets$x~~$5\gets$y}{$2\gets$Democrat(x)\\ 
  $5\gets$proposal\_for\_reform(y)\\ $2\gets$advocate(x,y)}{
  $2<5$}$\Rightarrow$\pdrs{3}{$3\gets$p}{$3\gets$says(x,p)\\ 
  $3\gets$p:\pdrs{4}{}{$6_S\gets$stupid\_thing(y)\\ 
  $4\gets$worthwhile(y)}{$4<6$~~$6=5$}}{$3<2$}}{}

}

\noindent The epithet ``the stupid thing'' is interpreted in \Last[a] as
contributing both an asserted, anaphoric referent for the adjective
``worthwhile'', and a subjective condition on this referent. By making use
of multiple pointers (which in terms of interpretation comes down to
duplicating the declaration of the referent and the condition), the  dual
role of the expressive noun phrase becomes explicit: it both provides an
antecedent for the asserted statement, and conveys the meaning of the
speaker about this antecedent. Note that the antecedent (``a proposal for
reform'') is itself not a projecting referent, but since the epithet serves
as an anaphor, the object of expression is projected, just as in the case of
supplemental CIs. 
%%explain better
The representation in \Last[b] (where the referent equality is eliminated)
represents the specific reading of the indefinite noun phrase ``a proposal
for reform''. It may be argued that the use of an expressive requires
(presupposes) a projecting anchor, since it only makes sense to express
feelings about some uniquely identifiable object. 

\noindent\parbox[b]{\textwidth}{\paragraph{Honorifics}
\ex. Ame~~ga~~~furi-\underline{mashi}-ta.\\
\textit{rain} \textsc{subj} \textit{fall}-\textsc{hon}-past\\
`It rained.' (\textit{performative honorific})\\
\pdrs{1}{$2\gets$e}{$2\gets$rain(e)\\ $1\gets$fell(e)\\
  $3_S\gets$\textsc{hon}(e)}{$1<2$~~$1<3$~~$3=2$}

}

\noindent The requirement for a projecting anchor is even clearer in the
case of honorifics, since they always serve as definite descriptions that
refer to a specific referent \citep[cf.][]{potts2004japanese}. Therefore,
the `raining event' in \Last is interpreted as a presupposition (referring
to the general concept of rain), so that the performative honorific can be
correctly interpreted by means of projection (again, the referential
equality is resolved).

\noindent\parbox[b]{\textwidth}{\paragraph{German Konjunktiv I} (lack of speaker commitment)
\ex. Sheila behauptet dass sie krank \underline{sei}.\\
\textit{Sheila maintains that she sick~~~be}-\textsc{konj}\\
`Sheila maintains/claims that she is sick.'\\
\pdrs{1}{$4\gets$x~~$1\gets$p}{$4\gets$Sheila(x)\\ $1\gets$behauptet(x,p)\\
  $1\gets$p:\pdrs{2}{}{$2\gets$krank(x)}{$2<4$}\\ 
  $5_S\gets\neg$\pdrs{3}{}{$3\gets$krank(x)}{$3<4$}}{$1<4$~~$1<5$~~$5=4$}

}

\subsubsection{Supplemental Adverbs}


Utterance level modifiers (e.g., ``Frankly, I don't like him'', ``In case
you're hungry, I'm going to the grocery.'') behave differently than other
supplemental CIs \cite[cf.][pp.725-729]{amaral2007review}.




\section{Discussion}


\section{Conclusion}


%\section{Section}
%\subsection{Subsection}
%\subsubsection{Subsubsection}
%\paragraph{Paragraph}
%\subparagraph{Subparagraph}
%
%\begin{table}
%  \begin{tabular}{c|c|c}
%     1 & 2 & 3 \\
%     la & di & da \\
%  \end{tabular}
%  \caption{An example of table}
%  \label{my_table}
%\end{table}

%=====================================================================

\newpage %%XXX comment out in final version
\bibliography{../../presupposition}

%=====================================================================

\begin{addresses}
  \begin{address}
    Noortje J. Venhuizen \\
    CLCG, University of Groningen\\
    Oude Kijk in 't Jatstraat 26\\
    9712 EK Groningen\\
    The Netherlands \\
    \email{n.j.venhuizen@rug.nl}
  \end{address}
  \begin{address}
    Johan Bos \\
    CLCG, University of Groningen\\
    Oude Kijk in 't Jatstraat 26\\
    9712 EK Groningen\\
    The Netherlands \\
    \email{johan.bos@rug.nl}
  \end{address}
  \begin{address}
    Petra Hendriks \\
    CLCG, University of Groningen\\
    Oude Kijk in 't Jatstraat 26\\
    9712 EK Groningen\\
    The Netherlands \\
    \email{p.hendriks@rug.nl}
  \end{address}
  \begin{address}
    Harm Brouwer \\
    CLCG, University of Groningen\\
    Oude Kijk in 't Jatstraat 26\\
    9712 EK Groningen\\
    The Netherlands \\
    \email{harm.brouwer@rug.nl}
  \end{address}
\end{addresses}

%=====================================================================

\clearpage

%XXX 
\bigskip
\noindent
\begin{tabular}{p{0.2\textwidth} p{0.75\textwidth}}
          \textbf{Goal:} & A unified account of projection phenomena.\\
   \textbf{Challenge I:} & CIs provide \textit{new} information, but they
                           project as if providing \textit{old} information
                           (like presuppositions).\\
   \textbf{Observation:} & Supplemental CIs occur with a projecting anchor,
                           and always project as far as their anchor.\\
    \textbf{Solution I:} & CIs provide \textit{new} information to the
                           (projecting) context created by a presupposition,
                           namely the one provided by its anchor.\\
\textbf{Implementation:} & CIs are projection-anaphoric in the sense that
                           they inherit their projection context from their
                           (projecting) anchor. This is formalized in PDRT
                           \citep{venhuizen2013iwcs} by introducing a
                           dependency at the lexical level between the
                           projection site of the CI and the projection site
                           of the anchor.\\
     \textbf{Next Step:} & Other CIs: expressives and supplemental adverbs.\\
  \textbf{Challenge II:} & Incorporating subject-orientedness in (P)DRT.\\
   \textbf{Solution II:} & Subjective PDRT: using pointers to restrict speaker
                           models.\\
    \textbf{Discussion:} & Are CIs a homogeneous class? What is their relation
                           to presuppositions and anaphora?\\
\end{tabular}\\

References: \citep{delgobbo2003appositives,potts2005logic,amaral2007review,
  nouwen2007appositives,harris2009perspective,heringa2012appositional,
  schlenker2013supplements,nouwen2014note,potts2013presupposition}

\end{document}
