\section{Discussion}

\subsection{Evaluating the analysis}

\subsection{Non-supplemental CIs: Expressives}

We focus on supplemental CIs since these are the most common and
well-studied type of CI.  However, a case can be made for a similar account
of expressives \citep[including expressive attributive adjectives, epithets,
honorifics, and tense-variations such as the German `Konjunktiv I';
cf.][]{potts2005logic}. The status of expressives has been open to some
debate \citep[...]{potts2004japanese,geurts2007fucking}.  But like
supplemental CIs, expressives in general require some antecedent (or:
anchor) to express some opinion about; this antecedent may have the form of
a noun phrase, an adjective, a proposition, or even a verb.
%%examples (see, e.g. Amaral et al. p. 711)
Each of these antecedents may result in different projection patterns, which
may be explained by the semantics of the different antecedents. However,
the notion of speaker-orientedness is especially important in this context,
so a framework is required that can incorporate speaker attitudes.

\subsection{Speaker-orientedness}

