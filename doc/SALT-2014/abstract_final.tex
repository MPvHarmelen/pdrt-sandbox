\documentclass[letterpaper,11pt]{article}

\usepackage{natbib}
\renewcommand\bibsection{\vspace{-0.25cm}\paragraph{\normalsize \refname}}
\usepackage{linguex}
\renewcommand\firstrefdash{}
\usepackage[top=0.7in, bottom=0.7in, left=1in, right=1in]{geometry}

\usepackage{amssymb}
\usepackage{verbatim}%for comment command
\usepackage{amsthm}
\usepackage{multicol}

%%%%% Boxes %%%%%

\newcommand{\drs}[2]
{\begin{tabular}{c}
  \noalign{\smallskip}
  \begin{tabular}{|l|}
    \hline #1\\ \hline #2\\ \hline
  \end{tabular}
  \smallskip
\end{tabular}
}

\newcommand{\pdrs}[3]
{\begin{tabular}{c}
  \noalign{\smallskip}
  $#1:$
  \begin{tabular}{|l|}
    \hline #2\\ \hline #3\\ \hline %\small#4 \\ \hline
  \end{tabular}
  \smallskip
\end{tabular}
}

\newcommand{\pdrsmap}[4]
{\begin{tabular}{c}
  \noalign{\smallskip}
  $#1:$
  \begin{tabular}{|l|}
    \hline #2\\ \hline #3\\ \hline \small #4 \\ \hline
  \end{tabular}
  \smallskip
\end{tabular}
}

\let\oldthebibliography=\thebibliography
  \let\endoldthebibliography=\endthebibliography
  \renewenvironment{thebibliography}[1]{%
    \begin{oldthebibliography}{#1}%
      \setlength{\parskip}{0ex}%
      \setlength{\itemsep}{0ex}%
  }{\end{oldthebibliography}}

\title{\vspace{-2em} How and Why Conventional Implicatures Project}
\date{}

\author{Noortje J. Venhuizen \and Johan Bos \and Petra Hendriks \and Harm Brouwer \and
  \texttt{\{n.j.venhuizen, johan.bos, p.hendriks, harm.brouwer\}@rug.nl}\\  
CLCG, University of Groningen
\vspace{-2em}}

%%%% Begin Document %%%%
\begin{document}
\normalsize
\maketitle

\pagestyle{empty}
\thispagestyle{empty}


%\section*{How and Why Conventional Implicatures Project}

\noindent 
%It is well-known that the meanings of utterances are structured into parts
%that behave differently with respect to how they are interpreted in the
%unfolding discourse.
It is well-known that a single utterance may convey different types of
information, for example \emph{new} and \emph{old} information. It is not
always clear, however, how these types of content relate to each other.
%Simons et al. \cite{simons2010projects} define content that does not
%address the Question Under Discussion
%\cite[cf.][]{roberts1996information-short} as \textit{not at-issue} (or
%\emph{off-issue}).  At-issueness affects the interpretation of semantic
%content. In particular, off-issue content tends to be unaffected by
%operators such as negation; it \emph{projects}. 
One way to pursue a unified semantic analysis would be to see how different
types of content that exhibit the same semantic property relate to each
other. Of particular interest in this context is the property of
\textit{projection}; the indifference of linguistic content to semantic
operators such as negation. The three most prominent classes of projection
phenomena are anaphora, presuppositions, and conventional implicatures (CIs;
as defined by Potts \cite{potts2005logic}) \cite{simons2010projects}.
%they are called like that because they are implicatures triggered by the
%conventional meaning of linguistic expressions.
The correspondence between the former two forms the basis for one of the
major accounts of presuppositions
\citep{sandt1992presupposition-short,geurts1999presuppositions}. On the
other hand, formal analyses of CIs have mainly focused on differentiating
between the semantic contribution of presuppositional and CI content by
introducing different meaning dimensions
\cite{potts2005logic,nouwen2007appositives-short}, or different types of
discourse updating \citep{anderbois2010crossing}. However, such accounts shed
little light on the commonalities underlying the projection behaviour that
is shared among presuppositions, anaphora, and CIs.

%% XXX andere overgang?
We propose a unidimensional and incremental analysis of conventional
implicatures, which highlights their correspondence to presuppositions, and
anaphora.  We focus on supplemental CIs, triggered by subordinated
constructions such as appositives, e.g., \emph{``John\underline{,
a linguist,} was not at the party''}. Our analysis is based on the
observations that (i) CIs always attach to an anchor, and (ii) the anchor
itself always projects, and (iii) CIs always project as far as their anchor.
These observations have led to several syntactocentric analyses for CIs
\cite[e.g.,][]{nouwen2007appositives-short,schlenker2013supplements,nouwen2014note},
but these fail to capture the \emph{semantic} properties underlying the
projection behaviour of CIs, presuppositions and anaphora. We therefore
propose a Montague-style semantic analysis that treats CIs as
`projection-anaphoric' (\emph{p-anaphoric}) to their anchor. This means that
CI content inherits the projection site from its anchor, while contributing
novel information to the discourse context. In contrast, presuppositions are
`reference-anaphoric' \citep[cf.][]{sandt1992presupposition-short}, which
signals previously established content and entails p-anaphoricity. The
analysis is formalized in Projective DRT \cite{venhuizen2013iwcs-short},
a representational framework in which projection sites are explicitly part
of the semantic representations.  The analysis explains the interpretational
differences between presuppositions and CIs, without stipulating
a fundamental distinction between them.

%\vspace{-0.4cm}
\paragraph{CIs are projection-anaphoric.} van der Sandt's
\citep{sandt1992presupposition-short} proposal to treat presupposition
projection as anaphora resolution, formalized in DRT \cite{kamp1981theory-short},
is based on the observation that presuppositions behave in a way similar to
anaphora. Conventional implicatures, on the other hand, are more similar to
regular assertions, since they are infelicitous in a context in which their
content has already been established, as in \emph{``John is a linguist.(...)
\#John, a linguist, ..''}. Like presuppositions, however, CIs are off-issue
and thus project out of embedded contexts. These characteristics can be
brought together, by treating CIs as \emph{projection-anaphoric} to their
anchor: their content is novel (i.e., non-anaphoric), but projects along
with the presupposition triggered by their anchor, thus contributing novel
information to the projection site of the anchor.  Moreover, since CIs are
non-restrictive, they require a specific, and therefore projecting anchor.
%e.g., in \emph{``If a professor, who is famous, comes to the party,..''}
%the indefinite must single out a specific professor since otherwise the
%relative clause will be restrictive. 
This explains why CIs are infelicitous when anchored to a non-specific
indefinite, as in \emph{\#``No man, a linguist,..''}. Thus, besides
`piggybacking' on their anchor, CIs require their anchor to project,
which explains why, like presuppositions, they tend to project globally.
In order to formalize this
behaviour, we need a framework in which projection sites are explicitly part
of the semantic representations.  This is part and parcel of Projective DRT
\cite{venhuizen2013iwcs-short}, where projection pointers indicate
the integration site of semantic content. 

%\vspace{-0.4cm} 
\paragraph{CIs in PDRT.} In Projective DRT, the correspondence between
anaphora and presuppositions is taken a step further by treating projection
as variable binding. Each context introduces a \emph{label} that can bind
the \emph{pointers} associated with the discourse referents and conditions,
indicating where the content is interpreted. To formalize p-anaphoricity, we
add structural information to PDRSs, via a subordination relation between
contexts \citep[cf.][]{reyle1993dealing-short}.
%: $A \leq B$ means that $A$ is subordinated to $B$. 
The projection of presuppositions is signalled by means of a strict
subordination ($<$).  CIs project as well, but also introduce an equality
between their projection site and the one provided by their anchor.  In
example \Next labels and pointers are indicated with a subscript, and
contextual constraints are shown after the conditions. 
%Example \Next[b] shows a presupposition embedded in a CI, which projects
%relative to the projection site of the CI itself.

\vspace{-0.25cm}
\ex.  Mary, a linguist, laughs.\\
{\small 
  $[~x_3,~z_2~
   |~Mary_3(x),~linguist_2(z),~x=_2z,~laugh_1(x)~
 |~1 < 3,~1 < 2,~2=3]_1$
}

\vspace{-0.25cm}
\noindent In contrast to a van der Sandtian analysis of CIs
based on variable trapping, PDRT predicts the infelicity of attaching a CI
to a non-projecting anchor. Due to the p-anaphoricity of CIs together with
their projecting nature, this results in conflicting contextual constraints,
as illustrated in example \Next, with the conflicting constraints
indicated in bold.

\vspace{-0.25cm}
\ex. \#No man, a linguist, laughs.\\
{\small 
  $[|\neg_1[~x_2,~z_3~
   |~man_2(x),~linguist_3(z),~x=_3z,~laugh_2(x)~
   |{\mathbf{~2 < 3,~3 = 2}},~2 < 1]_2|]_1$
}

\vspace{-0.25cm} 
\noindent An incremental construction procedure is crucial
for a proper account of CI content, because of its interaction with asserted
content, e.g., via anaphoric dependencies \citep{anderbois2010crossing}.  In
PDRT, this incremental construction can be formalized using Montague
semantics \citep[cf.][]{muskens1996combining}.  Importantly, since
projection is directly part of the incremental construction, and not treated
as a post-hoc mechanism \citep[as in][]{sandt1992presupposition-short},
anaphoric dependencies as well as conflicting contextual constraints are
directly available during discourse construction. To test and evaluate the
non-trivial procedure of dealing with projection variables during discourse
construction, we have implemented PDRT as a Haskell library, called
\textsc{pdrt-sandbox}.
%providing a platform for testing and evaluating the PDRT framework, as well
%as its predictions with respect to the behaviour of CIs

%We have implemented the construction
%procedure of PDRSs, as well as their translation to DRT and FOL, in a
%Haskell library called \textsc{pdrt-sandbox}, which treats unresolved
%structures as pure Haskell functions. This provides a platform to evaluate
%the framework, as well as its predictions with respect to the behaviour of CIs. 

%\vspace{-0.4cm} 
\paragraph{Toward a unified account of projected content.} Formal approaches
to semantics aim to capture all aspects of the meaning of an utterance. This
means incorporating different types of content (for example, \emph{old}
versus \emph{new}), while taking into account the interaction between them.
We provide a first step toward such a unified analysis of projected content.
By treating CIs as projection-anaphora, we can explain their kinship to
presuppositions and anaphora, without introducing an extra level of
complexity to account for the fact that they introduce novel information.

\small
\vspace{0.4cm}
\bibliographystyle{abbrv} 
\bibliography{../../presupposition} 
\end{document}
