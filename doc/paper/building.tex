\section{Building Projective Discourse Representation Structures}
\label{sec:building}

% nice introduction

\subsection{Basic Structures}

The box representations shown in Section~\ref{sec:preliminaries} are an
intuitive way to look at DRSs and PDRSs, but less useful for formal
reasoning about these structures. For this, we adhere to the set-theoretical
underpinnings of the formalism as described by
\citeasnoun{kamp1993discourse}, without their duplex conditions in order to
allow for a direct translation to first-order logic (see
Section~\ref{sec:translations}). This definition basically follows the one
proposed by \citeasnoun{bos2003implementing}, with the exception that
equality between variables (i.e., $x_1=x_2$) is not a separate DRS
condition, but is treated as a variant of a 2-place predicate (i.e.,
$R(x_1,x_2)$, where $R$ equals $=$). The definition of a basic DRS then
looks as follows:

\begin{definition}[Basic DRS] \label{def:bDRS}~\\
A Basic DRS is a tuple $\langle \{x_1 ... x_n\},\{\gamma_1 ... \gamma_m\} 
\rangle$, where:
 \begin{enumerate}[i.]
  \item $\{x_1 ... x_n\}$ is a finite set of variables;
  \item $\{\gamma_1 ... \gamma_m\}$ is a finite set of DRS conditions (which
    may be either basic or complex);
  \item\label{def:bDRS:Rel} $R(x_1, ..., x_n)$ is a basic DRS condition,
    with $x_1 ... x_n$ are variables and $R$ is a relation symbol for an
    $n$-place predicate;
  \item $\neg K$, $\Box K$ and $\Diamond K$ are complex DRS conditions, with
    $K$ is a DRS;
  \item $K_1 \vee K_2$ and $K_1 \Rightarrow K_2$ are complex DRS conditions,
    with $K_1$ and $K_2$ are DRSs;
  \item \label{def:bDRS:Prop} $x:K$ is a complex DRS condition, with $x$ is
    a variable and $K$ is a DRS.
 \end{enumerate} 
\end{definition}

\noindent The variables in a DRS are also called \textit{DRS referents} or
\textit{discourse referents}, and the set of referents is called the
\textit{universe} of the DRS.  There are several representations that can be
used to represent a DRS besides its set-theoretical representation. The most
widely used representation is the box representation already illustrated in
section~\ref{sec:preliminaries}, but for reasons of space we will also
sometimes use the linear box representation. Example~\Next shows the three
different representations for one sentence:

\ex. John is a vegetarian.
\a. $\langle \{x, y\},\{$John$(x),$vegetarian$(y), =(x,y)\}\rangle$
\b. \drs{$x~~y$}{John($x$)\\ vegetarian($y$)\\ $x=y$}
\c. \flatdrs{$x, y$}{John($x$), vegetarian($y$), $x=y$}

Moving to the theory of PDRT, the main difference between a basic DRS and
a basic PDRS is that in PDRT all structures introduce a label, which can
accordingly bind the pointers associated with the PDRS referents and
conditions.  Moreover, as described in Section~\ref{sec:pdrt}, the contexts
referred to by the pointers may introduce various dependencies, which is
reflected in a set of minimally accessible projection contexts, or \MAPs~for
short.  Thus, in set-theoretic terms, a basic PDRS is a quadruple that
consists of a label, a set of \MAPs, a set of projected referents (i.e., DRS
referents associated with a pointer), and a set of projected conditions.
This is formalized below:

\begin{definition}[Basic PDRS] \label{def:bPDRS}~\\
A Basic PDRS is a quadruple $\langle \rho, \{\mu_1 ... \mu_n\}, 
\{\delta_1 ... \delta_m\}, \{\chi_1 ... \chi_l\}\rangle$, where:
  \begin{enumerate}[i.]
    \item $\rho$ is a projection variable;
    \item $\{\mu_1 ... \mu_n\}$ is a finite set of \MAPs, with $\mu_i=\langle
      v_1,v_2\rangle$, and  $v_1$ and $v_2$ are projection variables;
    \item $\{\delta_1 ... \delta_m\}$ is a finite set of projected
      referents, with $\delta_j=\langle v_j, x_j\rangle$, such that $v_j$ is
      a projection variable, and $x_j$ is a DRS referent;
    \item $\{\chi_1 ... \chi_l\}$ is a finite set of projected conditions,
      with $\chi_k = \langle v_k,\gamma_k\rangle$, such that $v_k$ is a
      projection variable, and $\gamma_k$ is a PDRS condition (which may be
      either basic or complex);
    \item \label{def:bPDRS:Rel} $R(x_1, ..., x_n)$ is a basic PDRS condition,
      with $x_1 ... x_n$ are variables and $R$ is a relation symbol for an
      $n$-place predicate;
    \item $\neg P$, $\Box P$ and $\Diamond P$ are complex PDRS conditions,
      with $P$ is a PDRS;
    \item $P_1 \vee P_2$ and $P_1 \Rightarrow P_2$ are complex PDRS
      conditions, with $P_1$ and $P_2$ are PDRSs;
    \item\label{def:bPDRS:Prop} $x:P$ is a complex PDRS condition, with $x$
      is a variable and $P$ is a PDRS;
  \end{enumerate}
\end{definition}

\noindent Note the direct correspondence between Definition
\ref{def:bPDRS}(\ref{def:bPDRS:Rel}-\ref{def:bPDRS:Prop}) and Definition
\ref{def:bDRS}(\ref{def:bDRS:Rel}-\ref{def:bDRS:Prop}); the only difference
is that complex PDRS conditions contain subordinated PDRSs, and complex DRS
conditions contain subordinated DRSs.  The definition of projected referents
and projected conditions as tuples allows us to talk about the two elements
of the tuples separately; we refer to the first element of all projected
referents and projected conditions as the \textit{pointer}, and the second
elements are called the \textit{PDRS referent} and \textit{PDRS condition},
respectively. This is useful, since in some cases we are interested in the
entire projected referent/condition, whereas in other cases we need only to
refer to one of the elements of the tuple. Similarly, we refer to the first
element of the quadruple representing a PDRS, as the \textit{label} of the
PDRS (formally, the label of a PDRS $P_i$ is represented as $\rho(P_i)$).
We will refer to labels and pointers together as \textit{projection
variables}.  Again, we can represent PDRSs using the same representations as
defined for DRSs: the set-theoretic representation, the box representation
and the linear representation, illustrated in \Next.

\ex. John is a vegetarian.
\a. $\langle 2,\{\langle 1, x\rangle, \langle 2, y\rangle\},\{\langle1, 
      $John$(x)\rangle,\langle2, $vegetarian$(y)\rangle, \langle 2, =(x,y)
      \rangle\},\{\langle 2,1\rangle\}\rangle$
\b. \pdrs{$2$}{$1\gets x~~2\gets y$}{$1\gets$John($x$)\\ 
      $2\gets$vegetarian($y$)\\ $2\gets x=y$}{$2\leq 1$}
\c. \flatpdrs{$2$}{$1\gets x, 2\gets y$}{$1\gets$John($x$), 
      $2\gets$vegetarian($y$), $2\gets x=y$}{$2\leq 1$}


\subsection{Subordination and Accessibility}

One of the central notions in DRT is the notion of accessibility between
DRSs. Accessibility determines which DRS universes are accessible from a DRS
condition in order to bind its referents. This is crucial for determining
for example which properties hold for some given referent.  Accessibility
between DRSs is standardly defined based on a subordination relation
($\leq$) between DRSs, i.e., DRS $K_1$ is accessible from DRS $K_2$
\textit{iff} $K_2\leq K_1$, where subordination is recursively defined as
follows:

\begin{definition}[DRS Subordination]\label{def:DRSsub}~\\
DRS $K_1$ subordinates DRS $K_2$ ($K_2 \leq K_1$) iff:
  \begin{itemize}
    \item $K_1 = K_2$;
    \item $K_1$ directly subordinates $K_2$ ($K_2 <!~K_1$);
    \item There is a DRS $K_{1'}$, such that $K_{1'} \leq K_1$ and 
      $K_2 \leq K_{1'}$.
  \end{itemize}
\end{definition}

\begin{subdefinition}[DRS Direct Subordination]~\\
DRS $K_i$ directly subordinates DRS $K_j$ ($K_j <!~K_i$) iff:
  \begin{itemize}
    \item $\neg K_j$, $\Box K_j$ or $\Diamond K_j$ is a DRS condition in
      $K_i$;
    \item $x:K_j$ is a DRS condition in $K_i$ for some $x$;
    \item $K_j \Rightarrow K_k$, $K_k \Rightarrow K_j$, $K_j \vee K_k$, or 
      $K_k \vee K_j$ is a DRS condition in $K_i$ for some DRS $K_k$;
    \item There is a DRS $K_k$, such that $K_i \Rightarrow K_j$
      is a DRS condition in $K_k$.
  \end{itemize}
\end{subdefinition}

\noindent Thus, a DRS is always subordinated by itself, and by any DRS that
directly, or indirectly subordinates it. Direct subordination between two
DRSs $K_1$ and $K_2$ means that $K_2$ is either part of a condition in
$K_1$, or that $K_1$ serves as its antecedent in an implication.

For the accessibility relations in PDRT we need to take into account the
contexts introduced by the projection pointers.  In case a pointer occurs
free, it introduces a projected context. Since the interpretation site of
some linguistic content should always be accessible from its introduction
site, the projected context is necessarily accessible from the introduction
context. Moreover, additional constraints on accessibility may be determined
by the \MAPs~in the PDRSs from which the projected contexts are accessible.
So, since we need to consider both sub-PDRSs and projected contexts, we
define accessibility over \textit{PDRS-contexts} instead of PDRSs.
A PDRS-context may refer to a PDRS (via its label), or to a projected
context created by a pointer. PDRS subordination is then defined as follows:

\begin{definition}[PDRS-context Subordination]\label{def:PDRSsub}~\\
PDRS-context $\pi_1$ subordinates PDRS-context $\pi_2$ ($\pi_2 \leq \pi_1$)
iff:
\begin{itemize}
  \item $\pi_1 = \pi_2$;
  \item $\pi_1$ directly subordinates $\pi_2$ ($\pi_2 <!~\pi_1$);
  \item $\pi_1$ is a minimally accessible context with respect to $\pi_2$
    ($\pi_2 \preceq \pi_1$);    
  \item There is a PDRS-context $\pi_{1'}$, such that $\pi_1$ subordinates
    $\pi_{1'}$ and $\pi_{1'}$ subordinates $\pi_2$.
\end{itemize}
\end{definition}

\begin{subdefinition}[PDRS-context Direct Subordination]~\\
PDRS-context $\pi_i$ directly subordinates PDRS-context $\pi_j$ 
($\pi_j <!~\pi_i$) iff:
  \begin{itemize}
    \item $\langle\pi_i,\neg P_j\rangle$,
      $\langle\pi_i,\Box P_j\rangle$,
      or $\langle\pi_i,\Diamond P_j\rangle$ is a projected condition in
      some PDRS $P_k$, and $\rho(P_j) = \pi_j$;
    \item $\langle\pi_i,x:P_j\rangle$ is a projected condition in some
      PDRS $P_k$, for some $x$, and $\rho(P_j) = \pi_j$;
    \item  $\langle\pi_i,P_j \Rightarrow P_{j'}\rangle$ or
      $\langle\pi_i,P_j \vee P_{j'}\rangle$ is a projected condition in
      some PDRS $P_k$, and $\rho(P_j) = \pi_j$ or $\rho(P_{j'}) = \pi_j$;
    \item There is a PDRS $P_k$, such that $P_i \Rightarrow P_j$ is a PDRS
      condition in $P_k$, and $\rho(P_i) = \pi_i$ and $\rho(P_j) = \pi_j$.
    %NB this is about PDRS condition instead of projected condition
  \end{itemize}
\end{subdefinition}

\begin{subdefinition}[Minimally Accessible PDRS-Contexts]~\\
PDRS-context $\pi_i$ is a minimally accessible PDRS-context with
respect to $\pi_j$ ($\pi_j \preceq \pi_i$) iff:
  \begin{itemize}
    \item $\pi_j = \pi_i$;
    \item There is some PDRS $P_k$, such that $\pi_i$ is introduced in $P_k$
      and $\rho(P_k)=\pi_j$;
    \item $\langle \pi_j,\pi_i\rangle$ is an element of the set of
      \MAPs~of some PDRS $P_k$;
      %Moet dit nog gekwalificeerd worden? Ziets als:
      %, such that $\rho(P_k) \leq \pi_j$, and either
      %$\rho(P_k) \leq \pi_i$, or $\pi_i$ occurs free. % check this def!
    \item There exists a $\pi_k$, such that $\pi_j \preceq \pi_k$ and $\pi_k \preceq \pi_i$.
      % identity and transitive closure now duplicate in subordination
      % definition, but necessary for using this sub-definition separately.
  \end{itemize}
\end{subdefinition}

%XXX

The $\leq$-relation creates a partial order over all PDRS-contexts. This
results in a directed graph, called the \textit{Accessibility Graph}, over
all PDRS-contexts in a given PDRS. We will call a PDRS whose accessibility
graph is weakly connected,\footnote{A directed graph $G$ is weakly connected
iff for all pairs of vertices ($v_1$,$v_2$) in $G$ it holds that if all
directed edges in $G$ are replaced by undirected edges, then there is a path
between $v_1$ and $v_2$.} a \emph{connected} PDRS. All basic PDRSs are
connected PDRss, because projection variables either occur free (in which
case they are accessible from their introduction site), or are bound by some
sub-PDRS, which is necessarily accessible from the global PDRS. We will see
that unresolved structures, introduced below, may result in accessibility
graphs that consist of multiple non-connected components.
%XXX unresolved Merges may result in a non-connected graph (but then watch
%out for accidental bindings!)
%%XXX Define properly in Sandbox 

%resolved PDRS (no projected contexts): accessibility graph becomes tree.

\subsection{Variable Binding}

Up to this point we have been talking informally about free and bound
referents and projection variables. DRT and PDRT are dynamic semantic
formalisms, which means that the meaning structures simultaneously provide
semantic content and serve as a context for novel information. That is, the
referents introduced in a (P)DRS may serve as antecedents for later
introduced anaphoric expressions. In DRT, a referent is bound in case it is
introduced in the universe of some accessible DRS, as formally defined below
(here, $U(K_i)$ indicates the set of referents in the universe of $K_i$).

\begin{definition}[DRS bound variable]~\\
DRS variable $x$, introduced in DRS $K_i$, is bound in global DRS $K$ iff:
\begin{quote}
There exists a DRS $K_j \leq K$, such that
\begin{enumerate}[i.]
  \item $K_i \leq K_j$;
  \item $x\in U(K_j)$. %, where $U(K_2)$ refers to the universe of $K_2$.
\end{enumerate}
\end{quote}
\end{definition}

\begin{definition}[DRS Free variables]~ %cf. K,vG&R(2005)
  \begin{enumerate}[i.]
    \item $FV(\langle U, C \rangle) = (\bigcup_{c\in C} FV(c)) - U$
    \item $FV(R(x_1,...,x_n)) = \{x_1,...,x_n\}$
    \item $FV(\neg K) = FV(\Diamond K) = FV(\Box K) = FV(K)$
    \item $FV(K_1 \Rightarrow K_2) = FV(K_1) \cup (FV(K_2) - U(K_1)$
    \item $FV(K_1 \Rightarrow K_2) = FV(K_1) \cup FV(K_2)$
    \item $FV(x:K) = \{x\} \cup FV(K)$
  \end{enumerate}
\end{definition}


Free and bound projection variables in PDRT are defined in the same way as
referents in DRT, except that the antecedent variable is required to
function as the label of some accessible PDRS:

\begin{definition}[PDRS bound projection variable]~\\
Projection variable $v$, introduced in PDRS $P_i$, is bound iff:
\begin{quote}
There exists a PDRS $P_j$, such that
\begin{enumerate}[i.]
  \item $\rho(P_i) \leq \rho(P_j)$; 
  \item $v = \rho(P_j)$.
\end{enumerate}
\end{quote}
\end{definition}

\begin{definition}[PDRS Free projection variables]~
  \begin{enumerate}[i.]
    \item $FPV(\langle l, M, U, C \rangle) = (\bigcup_{m\in M} FPV(m) \cup \bigcup_{u\in U} FPV(u) \cup \bigcup_{c\in C} FPV(c)) - \{l\}$
    \item $FPV(\langle p_1, p_2\rangle) = \{p_1, p_2\}$
    \item $FPV(\langle p, r\rangle) = \{p\}$
    \item $FPV(\langle p, R(...)\rangle) = \{p\}$
    \item $FPV(\langle p,\neg K\rangle) = FPV(\langle p,\Diamond K\rangle) = FPV(\langle p,\Box K\rangle) = FPV(\langle p,x:K\rangle) = \{p\} \cup FPV(K)$
    \item $FPV(\langle p,K_1 \Rightarrow K_2\rangle) = \{p\}\cup FPV(K_1) \cup (FPV(K_2) - \rho(K_1))$
    \item $FPV(\langle p,K_1 \vee K_2\rangle) = \{p\}\cup FPV(K_1) \cup FPV(K_2)$
  \end{enumerate}
\end{definition}


In PDRT, the interpretation site of the semantic content, indicated by the
pointer, determines whether some referent is bound. Again, a referent is
bound if it occurs in some universe as an accessible antecedent. However,
since in PDRT all referents and conditions are associated with a pointer,
and may therefore be projected, the accessibility of antecedents should not
be determined on the basis of the PDRS in which a referent is introduced,
but by where it is interpreted. Thus, a projected referent is bound in case
there exists some referent pointing to an accessible context in any universe
in the global PDRS. Since by definition the interpretation context of some semantic
content is accessible from its context of introduction, it follows from this
definition that the projection context of the antecedent is also accessible
from the introduction site of the projected referent ($\rho(P_i) \leq
\pi_j$). 

\begin{definition}[PDRS bound projected referent]~\\
Projected referent $\langle p,r\rangle$, introduced in PDRS $P_i$, is bound
in global PDRS $P$ iff:
\begin{quote}
There exists a PDRS-context $\pi_j \in PVars(P)$, such that
\begin{enumerate}[i.]
  \item $p \leq \pi_j$; 
%  \item $\rho(P_i) \leq \pi_j$; 
    %XXX is this by definition true, since $\rho(P_i) \leq p$ (given a
    %connected PDRS)?
  \item There exists some PDRS $P_j$, such that 
    $\langle \pi_j,r\rangle \in U(P_j)$.
\end{enumerate}
\end{quote}
\end{definition}

\noindent In this definition, the set \textit{PVars(P)} represents the set
of all projection variables (labels and pointers) that occur in PDRS $P$.

%%XXX

The set of \textit{free projected referents} of a PDRS $P$ are defined as the
difference between the \textit{possibly free projected referents} ($\mathcal{F}$) of
$P$ and the \textit{possibly bound projected referents} ($\mathcal{B}$) of $P$, based
on an empty set of projected referents and an empty set of \MAPs, thus:

\begin{definition}[PDRS Free Projected Referents]~
  \begin{itemize}
    \item $FPR(P) = \mathcal{F}(P) - \mathcal{B}(P,\emptyset,\emptyset)$.
  \end{itemize}
\end{definition}

We define the \textit{possibly free projected referents} of a PDRS $P$ on the
basis of the conditions of $P$: those referents that occur in a predicate or
a propositional are, together with the pointer of their respective
conditions, candidates to be free projected referents. The collection of
these referents can be described as follows:

\begin{subdefinition}[Possibly Free Projected Referents]~
  \begin{enumerate}[i.]
    \item $\mathcal{F}(\langle l, M, U, C \rangle)
      = (\bigcup_{c\in C} \mathcal{F}(c))$
    \item $\mathcal{F}(\langle p, R(x_1,...,x_n)\rangle)
      = \{\langle p, x_1\rangle, ..., \langle p, x_n\rangle\}$
    \item $\mathcal{F}(\langle p,\neg K\rangle) 
      = \mathcal{F}(\langle p,\Diamond K\rangle) 
      = \mathcal{F}(\langle p,\Box K\rangle)
      = \mathcal{F}(K)$
    \item $\mathcal{F}(\langle p,K_1 \Rightarrow K_2\rangle)
      = \mathcal{F}(\langle p,K_1 \vee K_2\rangle)
      = \mathcal{F}(K_1) \cup \mathcal{F}(K_2)$
    \item $\mathcal{F}(\langle p,x:K\rangle)
      = \{\langle p,x \rangle\} \cup \mathcal{F}(K)$
  \end{enumerate}
\end{subdefinition}

In order to determine the set of \textit{possibly bound referents} of PDRS
$P$, that may bind the possibly free projected referents, it is not enough
to just take the union of all universes in a PDRS. This is of course because
projected referents may bind other referents in any context from which their
projection site is accessible. In order to account for all these
hypothetical bindings, we collect not only the projected referents actually
occurring in the universe of $P$, but also the combinations of the referents
with all contexts from which their projection context is accessible. Since
the availability of accessible contexts is not hierarchical (i.e., the
projection context that binds a referent may be introduced in some embedded
PDRS), we need to keep track of the entire accessibility graph in order to
create all possible bindings; we can do this by keeping track of the
\MAPs~of all superordinate PDRSs and create possible bindings based on these
accessibilities. The collection and generation of possible bound referents
is shown below.  Informally, the algorithm works as follows: all previously
found referents, plus the universe of the current PDRS are collected, and
these are enhanced with hypothetical antecedents created on the basis of the
current label, and the current set of \MAPs~(by adding new projected
referents for those referents whose pointer occurs as the second element of
a \textsc{map}). In addition, all universes and hypothetical referents of
the PDRSs embedded in the conditions are collected, based on the referents
universes and \MAPs~obtained so far. Note that projected conditions (i.e.,
those conditions $c$ for which it holds that $c \notin Acc(C)$) are not
accessible to the bound referents obtained so far, and therefore do not take
these into account for creating possible bound projection referents.

\begin{subdefinition}[Possibly Bound Projected Referents]~\\
  The possibly bound projected referents are determined on the basis of a
  set of projected referents ($R$), and a set of \MAPs~representing
  accessibility between contexts ($A$).
  \begin{enumerate}[i.]
    \item $\mathcal{B}(\langle l, M, U, C \rangle,R,A) 
      = R' \cup (\bigcup_{\langle p, r\rangle\in R'} \langle l, r\rangle)
      \cup (\bigcup_{\langle p_1,p_2\rangle\in A'} R'[p_2\backslash p_1])
      \cup (\bigcup_{c\in Acc(C)} \mathcal{B}(c,R',A'))
      \cup (\bigcup_{c\notin Acc(C)} \mathcal{B}(c,\emptyset,\emptyset))$\\
      where: $R' = R \cup U$;
             $A' = A \cup M$;
             $Acc(C) = \{\langle p, c'\rangle \in C~|~ p \preceq l\}$
    \item $\mathcal{B}(\langle p, R(...)\rangle,R,A) = \emptyset$
    \item $\mathcal{B}(\langle p,\neg K\rangle,R,A)
      = \mathcal{B}(\langle p,\Diamond K\rangle,R,A)
      = \mathcal{B}(\langle p,\Box K\rangle,R,A)
      = \mathcal{B}(\langle p,x:K\rangle,R,A)
      = \mathcal{B}(K,R,A)$
    \item $\mathcal{B}(\langle p,K_1 \Rightarrow K_2\rangle,R,A)
      = \mathcal{B}(K_2,\mathcal{B}(K_1,R,A,),A\cup \MAPs(K_1))$
    \item $\mathcal{B}(\langle p,K_1 \vee K_2\rangle,R,A)
      = \mathcal{B}(K_1,R,A) \cup \mathcal{B}(K_2,R,A)$
      \end{enumerate}
\end{subdefinition}


Now we can define several properties:

\begin{definition}[Properness]~\\
A DRS $K$/PDRS $P$ is proper iff:
\begin{itemize}
  \item $K$/$P$ does not contain any free referents: $FV(K) = \emptyset$ / $FPR(P) = \emptyset$.
\end{itemize}
\end{definition}


\begin{definition}[Simpleness]~\\
A PDRS $P$ is simple (i.e., non-presuppositional) iff
\begin{itemize}
  \item $P$ does not contain any free pointers: $FPV(P) = \emptyset$.
\end{itemize}
\end{definition}


\begin{definition}[Pureness]~\\
A DRS $K$/PDRS $P$ is pure iff:
\begin{itemize}
  \item $K$/$P$ does not contain any otiose declarations of (projection) variables
    (i.e., $K$/$P$ does not contain any unbound, duplicate uses of variables).
    \begin{itemize}
      \item For all $K_1$, $K_2$, such that $K_2 \leq K_1 \leq K$: 
        $U(K_2) \cap (U(K_1) \cup FV(K)) = \emptyset$
      \item For all $P_1$, $P_2$, such that $P_2 \leq P_1 \leq K$: 
        $\{\rho(P_2)\} \cap (\{\rho(P_1)\} \cup FPR(P)) = \emptyset$
    \end{itemize}
\end{itemize}
\end{definition}


