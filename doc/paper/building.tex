\section{Building Projective Discourse Representation Structures}

% nice introduction

\subsection{Basic Structures and Subordination}

We define basic DRSs following \citeasnoun{bos2003implementing}, with the
exception that equality between variables (i.e., $x_1=x_2$) is not
a separate DRS condition, but is treated as a variant of a 2-place predicate
(i.e., $R(x_1,x_2)$, where $R$ equals $=$). This definition basically
follows the definition proposed by \citeasnoun{kamp1993discourse}, without
their duplex conditions, in order to allow for a direct translation to
first-order logic.

\begin{definition}[Basic DRS] \label{def:bDRS}
A Basic DRS is a tuple $\langle \{x_1 ... x_n\},\{\gamma_1 ... \gamma_m\} 
\rangle$, where:
 \begin{enumerate}[i.]
  \item $\{x_1 ... x_n\}$ is a finite set of variables (called ``DRS
    referents'');
  \item $\{\gamma_1 ... \gamma_m\}$ is a finite set of DRS conditions (which
    may be either basic or complex);
  \item\label{def:bDRS:Rel} $R(x_1, ..., x_n)$ is a basic DRS condition,
    with $x_1 ... x_n$ are variables and $R$ is a relation symbol for an
    $n$-place predicate;
  \item $\neg K$, $\Box K$ and $\Diamond K$ are complex DRS conditions, with
    $K$ is a DRS;
  \item $K_1 \vee K_2$ and $K_1 \Rightarrow K_2$ are complex DRS conditions,
    with $K_1$ and $K_2$ are DRSs;
  \item \label{def:bDRS:Prop} $x:K$ is a complex DRS condition, with $x$ is
    a variable and $K$ is a DRS.
 \end{enumerate} 
\end{definition}

% set of referents is DRS universe
%%more explanation?

The main difference between a basic DRS and a basic PDRS is that in PDRT all
structures introduce a label, which can accordingly bind the pointers
associated with the PDRS referents and conditions. We will refer to labels
and pointers together as \textit{projection variables}. This description
results in the definition of a PDRS as a triple of a projection variable,
a set of projected referents (i.e., DRS referents associated with a pointer)
and a set of projected conditions. However, such a definition omits an
essential part of the information that is required to properly represent
projection phenomena, namely their \emph{projection path}. In DRT, the
projection path (or ``projection line'', as
\citeasnoun{sandt1992presupposition} calls it) of some projected content
determines which DRSs are accessible from the introduction site to find
a possible antecedent in or otherwise to accommodate into. Accessibility
between DRSs is standardly defined based on a subordination relation
($\leq$) between DRSs, i.e., DRS $K_1$ is accessible from DRS $K_2$
\textit{iff} $K_2\leq K_1$, where subordination is defined as follows:

\begin{definition}[DRS Subordination]\label{def:DRSsub}
DRS $K_1$ subordinates DRS $K_2$ ($K_2 \leq K_1$) iff:
  \begin{itemize}
    \item $K_1 = K_2$;
    \item $K_1$ directly subordinates $K_2$;
    \item There is a DRS $K_{1'}$, such that $K_1$ subordinates $K_{1'}$ and 
      $K_{1'}$ subordinates $K_2$.
  \end{itemize}
  where: DRS $K_i$ directly subordinates DRS $K_j$ ($K_j <!~K_i$) iff:
  \begin{itemize}
    \item $\neg K_j$, $\Box K_j$ or $\Diamond K_j$ is a DRS condition in
      $K_i$;
    \item $x:K_j$ is a DRS condition in $K_i$ for some $x$;
    \item $K_j \Rightarrow K_k$, $K_k \Rightarrow K_j$, $K_j \vee K_k$, or 
      $K_k \vee K_j$ is a DRS condition in $K_i$ for some DRS $K_k$;
    \item There is a DRS $K_k$, such that $K_i \Rightarrow K_j$ or $K_i \vee K_j$
      is a DRS condition in $K_k$.
  \end{itemize}
\end{definition}

In PDRT, projection pointers refer to contexts that may or may not be part
of the current discourse structure. They are, however, related to the PDRS
in which they are introduced via a subordination relation, since the
interpretation site of some linguistic content is necessarily accessible
from its introduction site. Moreover, projected contexts may subordinate
other projected contexts, which happens for example in the case of embedded
presuppositions. So, in order to properly define the projection path of
projected content in PDRT, the subordination properties of projected
contexts should be explicitly part of the PDRS. We do this by enhancing each
PDRS with a set of \MAPs: Minimally Accessible Projected contexts, which are
basically tuples describing subordination pairs of contexts. This results in
the following definition of basic PDRSs:

\begin{definition}[Basic PDRS] \label{def:bPDRS}
A Basic PDRS is a quadruple $\langle \rho, \{\mu_1 ... \mu_l\}, 
\{\delta_1 ... \delta_n\}, \{\chi_1 ... \chi_m\}\rangle$, where:
  \begin{enumerate}[i.]
    \item $\rho$ is a projection variable;
    \item $\{\mu_1 ... \mu_l\}$ is a finite set of \MAPs, with $\mu_k=\langle
      v_1,v_2\rangle$, such that $v_1$ and $v_2$ are projection variables,
      and $v_2\leq v_1$.
    \item $\{\delta_1 ... \delta_n\}$ is a finite set of projected
      referents, with $\delta_i=\langle v_i, x_i\rangle$, such that $v_i$ is
      a projection variable, and $x_i$ is a DRS referent;
    \item $\{\chi_1 ... \chi_m\}$ is a finite set of projected conditions,
      with $\chi_j = \langle v_j,\gamma_j\rangle$, such that $v_j$ is a
      projection variable, and $\gamma_j$ is a PDRS condition (which may be
      either basic or complex);
    \item \label{def:bPDRS:Rel} $R(x_1, ..., x_n)$ is a basic PDRS condition,
      with $x_1 ... x_n$ are variables and $R$ is a relation symbol for an
      $n$-place predicate;
    \item $\neg P$, $\Box P$ and $\Diamond P$ are complex PDRS conditions,
      with $P$ is a PDRS;
    \item $P_1 \vee P_2$ and $P_1 \Rightarrow P_2$ are complex PDRS
      conditions, with $P_1$ and $P_2$ are PDRSs;
    \item\label{def:bPDRS:Prop} $x:P$ is a complex PDRS condition, with $x$
      is a variable and $P$ is a PDRS;
  \end{enumerate}
\end{definition}

\noindent Note the direct correspondence between Definition
\ref{def:bPDRS}(\ref{def:bPDRS:Rel}-\ref{def:bPDRS:Prop}) and Definition
\ref{def:bDRS}(\ref{def:bDRS:Rel}-\ref{def:bDRS:Prop}); the only difference
is that complex PDRS conditions contain subordinated PDRSs, and complex DRS
conditions contain subordinated DRSs.  The definition of projected referents
and projected conditions as tuples allows us to talk about the two elements
of the tuples separately; we refer to the first element of all projected
referents and projected conditions as the \textit{pointer}, and the second
elements are called the \textit{PDRS referent} and \textit{PDRS condition},
respectively. This is useful, since in some cases we are interested in the
entire projected referent/condition, whereas in other cases we need only to
refer to one of the elements of the tuple. Similarly, we refer to the first
element of the quadruple representing a PDRS, as the \textit{label} of the
PDRS (formally, the label of a PDRS $P_i$ is represented as $\rho(P_i)$).

Now we can define subordination for PDRSs. As described above, we need to
consider both sub-PDRSs and projected contexts when defining accessibility.
Therefore, we define accessibility over \textit{PDRS-contexts} instead of
PDRSs. A PDRS-context may refer to a PDRS (via its label), or to
a projected context created by a pointer.

\begin{definition}[PDRS-context Subordination]
PDRS-context $\pi_1$ subordinates PDRS-context $\pi_2$ ($\pi_2 \leq \pi_1$)
iff:
\begin{itemize}
  \item $\pi_1 = \pi_2$;
  \item $\langle \pi_2,\pi_1\rangle$ is an element of the set of \MAPs~of some PDRS $P_i$;
  \item $\pi_1$ directly subordinates $\pi_2$;
  \item There is a PDRS-context $\pi_{1'}$, such that $\pi_1$ subordinates
    $\pi_{1'}$ and $\pi_{1'}$ subordinates $\pi_2$.
\end{itemize}
where: PDRS-context $\pi_i$ directly subordinates PDRS-context $\pi_j$ 
($\pi_j <!~\pi_i$) iff:
\begin{itemize}
  \item $\langle\pi_i,\neg P_j\rangle$,
    $\langle\pi_i,\Box P_j\rangle$,
    or $\langle\pi_i,\Diamond P_j\rangle$ is a projected condition in
    some PDRS $P_k$, and $\rho(P_j) = \pi_j$;
  \item $\langle\pi_i,x:P_j\rangle$ is a projected condition in some
    PDRS $P_k$, for some $x$, and $\rho(P_j) = \pi_j$;
  \item  $\langle\pi_i,P_j \Rightarrow P_{j'}\rangle$ or
    $\langle\pi_i,P_j \vee P_{j'}\rangle$ is a projected condition in
    some PDRS $P_k$, and $\rho(P_j) = \pi_j$ or $\rho(P_{j'}) = \pi_j$;
  \item There is a PDRS $P_k$, such that $P_i \Rightarrow P_j$ or 
    $P_i \vee P_j$ is a PDRS condition in $P_k$, and $\rho(P_i) = \pi_i$ 
    and $\rho(P_j) = \pi_j$. 
    %NB this is about PDRS condition instead of projected condition
\end{itemize}
\end{definition}

\noindent Note that direct subordination for PDRS-contexts is not defined on
the basis of the set of conditions of some PDRS, as is the case for DRS
subordination in Definition~\ref{def:DRSsub}, but on the basis of the
pointer associated with the PDRS condition in which the sub-PDRS occurs.
This is because pointers indicate where some content is interpreted, which
is relevant for determining accessibility between contexts. This means that
some sub-PDRS that is part of the conditions of another PDRS, need not be
subordinated by it. 

The $\leq$-relation creates a partial order over all PDRS-contexts, provided
that subordination is defined for all projected contexts via the \MAPs. This
results in a directed graph, called the \textit{Accessibility Graph}, over
all PDRS-contexts in a given PDRS. We will call a PDRS whose accessibility
graph is weakly connected,\footnote{A directed graph $G$ is weakly connected
  iff for all pairs of vertices ($v_1$,$v_2$) in $G$ it holds that if all
  directed edges in $G$ are replaced by undirected edges, then there is
a path between $v_1$ and $v_2$.} a \emph{complete} PDRS. We will assume that
all PDRSs are complete throughout the rest of this paper.
%XXX Can we/Do we want to allow non-complete PDRSs in this sense? Isn't it
%the case that a free pointer necessarily subordinates its context of
%introduction? Should this be part of the definition of subordination? But:
%it seems like this might result in a circular definition, where
%accessibility depends on free pointers, and free pointers on accessibility.
%XXX Define properly in Sandbox 

%resolved PDRS (no projected contexts): accessibility graph becomes tree.

\subsection{Variable binding}

Now we are ready to define free and bound variables.  This is necessary to
define projection, anaphoric binding, PDRS renaming, etc.

\begin{definition}[DRS bound variables]
DRS variable $x$, introduced in DRS $K_i$, is bound iff:
\begin{quote}
There exists a DRS $K_j$, such that
\begin{enumerate}[i.]
  \item $K_i \leq K_j$;
  \item $x\in U(K_j)$. %, where $U(K_2)$ refers to the universe of $K_2$.
\end{enumerate}
\end{quote}
\end{definition}

\begin{definition}[PDRS bound projection variables]
Projection variable $v$, introduced in PDRS $P_i$, is bound iff:
\begin{quote}
There exists a PDRS $P_j$, such that
\begin{enumerate}[i.]
  \item $\rho(P_i) \leq \rho(P_j)$; %remember that subordination for PDRSs was defined over PDRS-contexts
  \item $v = \rho(P_j)$.
\end{enumerate}
\end{quote}
\end{definition}

\begin{definition}[PDRS bound projected referents]
Projected referent $\langle p,r\rangle$, introduced in PDRS $P_i$, is bound iff:
\begin{quote}
There exists a PDRS $P_j$, such that
\begin{enumerate}[i.]
  \item $p \leq \rho(P_j)$; 
  \item $\rho(P_i) \leq \rho(P_j)$; %XXX is this by definition true, since $P_i \leq p$ (in a complete PDRS)?
  \item $\langle p,r\rangle \in U(P_j)$.
\end{enumerate}
\end{quote}
\end{definition}



\subsection{Combining Structures}

\subsection{Unresolved Structures}
