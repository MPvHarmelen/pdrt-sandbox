\section{Combining Structures}

In the previous section we have defined basic structures and their related
properties, including context accessibility and bound variables.  As the aim
of a semantic formalism is to provide a framework for constructing meaning
representations, the logical next step is to define a way to combine
structures in order to create larger meaning representations. In this
section, we first describe how complete structures can be merged together,
and then introduce unresolved structures in order to create meaning
representations for smaller building blocks, like words.

\subsection{Merging}

In DRT, combining two DRS is defined straightforwardly as taking the union
of the sets of referents and conditions from both DRSs. One important
condition on combing structures is that no accidental binding of variables
occurs. To see this, consider the following example:

\ex. John smiles. Mary cries.\\
  \drs{x}{John(x)\\ smiles(x)} $\Cup$ \drs{x}{Mary(x)\\cries(x)} = 
  \drs{x}{John(x)\\ smiles(x)\\ Mary(x)\\ cries(x)}

\noindent This example illustrates that simply taking the set-theoretic
union of two DRSs may result in the wrong interpretation, since the
resulting DRS in \Last only introduces one referent, with four properties,
whereas the desired interpretation introduces two referents with each two
properties. In order to overcome this, we define a renaming function
$\mathcal{R}(K_1,K_2)$, which renames all (bound) variables in DRS $K_1$
which overlap with any variables occurring in $K_2$. %\footnote{The renaming
%function $\mathcal{R}$ is simply $\alpha$-conversion based on a fixed set
%of variables.} 
Since only bound variables are renamed, this function still allows for
anaphoric binding between DRSs, since variables that occur free will not be
renamed and may thus become bound. The resulting definition for DRS Merge is
shown below:

\begin{definition}[DRS Merge]\label{def:DRSmerge}~\\
Given DRSs $K_i=\langle U_i,C_i \rangle$ and $K_j=\langle U_j,C_j \rangle$,
the DRS merge between $K_i$ and $K_j$ ($K_i + K_j$) is defined as follows:
  \begin{quote}
    $K_i + K_j = 
    \langle U_i\cup U_{j'},C_i \cup C_{j'}\rangle$
  \end{quote}
  where: $K_{j'} = \langle U_{j'},C_{j'} \rangle = \mathcal{R}(K_j,K_i)$
\end{definition}

In PDRT, the projection variables should be taken into account when
combining structures.  In line with \citeasnoun{venhuizen2013iwcs} we can
define different types of merge for asserted and projected
content.\footnote{We here leave out the third merge for CI content described
in \citeasnoun{venhuizen2013iwcs}, since we have recently defined a new
analysis for CIs in PDRT that does not adhere to a special type of merge
\cite{venhuizenInPrepSALT}.} More specifically, ``assertive merge'' is an
operation between two PDRSs $P_1$ and $P_2$, which results a DRS in which
the asserted content of both $P_1$ and $P_2$ remains asserted. This is
effectuated by replacing all occurrences of the label of $P_1$ in the
universe and conditions of $P_1$ by the label of the resulting PDRS, that
is, by the label of (the renamed version of) $P_2$. The definition for
assertive merge is shown below. Here $S[v_1\backslash v_2]$ means that all
occurrences of $v_1$ in $S$ are replaced by $v_2$. Again, the renaming
function $\mathcal{R}$ renames all bound variables (projection variables and
DRS referents) in the first PDRS that overlap with variables in the second
PDRS.

\begin{definition}[PDRS Assertive Merge]\label{def:amerge}~\\
Given PDRSs $P_i=\langle \rho_i,M_i,U_i,C_i \rangle$ and
$P_j=\langle \rho_j,M_j,U_j,C_j \rangle$, the assertive merge between $P_i$
and $P_j$ ($P_i + P_j$) is defined as follows:
  \begin{quote}
    $P_i + P_j = 
      \langle \rho_{j'}, 
      M_{i}[\rho_i\backslash\rho_{j'}] \cup M_{j'},
      U_{i}[\rho_i\backslash\rho_{j'}] \cup U_{j'},
      C_{i}[\rho_i\backslash\rho_{j'}] \cup C_{j'}\rangle$
  \end{quote}
  where: $P_{j'} = \langle \rho_{j'}, M_{j'}, U_{j'}, C_{j'} \rangle =$
      $\mathcal{R}(P_j,P_i)$
\end{definition}

\noindent The operation for assertive merge results in a PDRS where all
content has the same status as it had before merging. However, we are
interested in creating a formalism in which some content may become
projected during discourse construction.  Intuitively, projected content
(and in particular presupposed content) has a status that is prior to
asserted content.  Therefore, we define ``projective merge'' as an operation
that results in the content of its first argument to become projected in the
resulting PDRS.  This operation is very similar to the $\alpha$-operator
introduced by \cite{bos2003implementing}, taking into account the projection
variables in a PDRS. Projection in PDRT means that the pointer of the
projected content indicates some accessible PDRS-context, which may either
be instantiated by a higher PDRS, or it may be a context that is projected
since the pointer is not bound by any label. By defining projective merge in
such a way that the projected content keeps its own pointer, we allow the
projection variables to become bound in an accessible PDRS, just as in the
case of anaphoric binding in DRT via the use of free variables that are not
renamed during merging. Since we want the result of the projective merge to
be a connected PDRS, the only addition to taking the set-theoretic union of
the content of the projected PDRS and the (renamed) context PDRS, is that
the resulting set of \MAPs~is enhanced with a relation expressing the
accessibility between the resulting PDRS and the context created through
projection, i.e., the label of the projected PDRS.

\begin{definition}[PDRS Projective Merge]\label{def:pmerge}~\\
Given PDRSs $P_i=\langle \rho_i,M_i,U_i,C_i \rangle$ and 
$P_j=\langle \rho_j,M_j,U_j,C_j\rangle$, the projective merge between $P_i$
and $P_j$ ($P_i \ast P_j$) is defined as follows:
  \begin{quote}
    $P_i \ast P_j = 
      \langle \rho_{j'}, 
        M_{i}\cup M_{j'}\cup\{\langle\rho_{j'},\rho_{i}\rangle\}, 
        U_{i}\cup U_{j'},C_{i} \cup C_{j'}\rangle$
  \end{quote}
  where: $P_{j'} = \langle \rho_{j'}, M_{j'}, U_{j'}, C_{j'} \rangle =$ 
      $\mathcal{R}(P_j,P_i)$
\end{definition}

\noindent In contrast to DRS Merge defined in Definition~\ref{def:DRSmerge},
the merge operations for PDRSs, defined in Definitions~\ref{def:amerge} and
\ref{def:pmerge} are non-symmetrical operations. This means that changing
the order of the arguments changes the outcome of the operation. In both
PDRS merges, the second PDRS functions as a kind of context in which the
content of the first PDRS is integrated, since the resulting PDRS will have
the label of the second PDRS. However, for the Assertive Merge
(Definition~\ref{def:amerge}) it is straightforward to show that changing
the order of the arguments results in an $\alpha$-equivalent PDRS, since the
only difference is that the projection variables and referents may be named
differently. This implies that the assertive merge operation is
interpretation-symmetrical, since the order of the arguments does not affect
the truth-values of the resulting PDRS. This, however, does not hold for
Projective Merge (Definition~\ref{def:pmerge}), since the status of the
content in the resulting PDRS is determined by the ordering of the merged
arguments: the asserted content of the first PDRS will be projected in the
resulting PDRS. This asymmetry results in a non-trivial interaction between
the two PDRS merges, represented in Table~\ref{tab:mergeinteractions}.

\begin{table}[h]
  \caption{Interaction between Assertive and Projective merge}
  \label{tab:mergeinteractions}
  \centering
  \begin{tabular}{| l c l |}
    \hline
    $A + (B + \Gamma)$ & $=$ & $(A + B) + \Gamma$\\
    \hline
    $A * (B * \Gamma)$ & $=$ & $(A * B) * \Gamma$\\
    \hline
    $A * (B + \Gamma)$ & $=$ & $(A * B) + \Gamma$\\
    \hline
    $A + (B * \Gamma)$ & $=$ & $B * (A + \Gamma) ~~=~~ (B * A) + \Gamma$\\
    \hline\hline
    $(A + B) * \Gamma$ & $=$ &  $(A + \mathcal{E}(B)) * (B * \Gamma)$\\
                       &&  \textit{where:} $\mathcal{E}(B) = \langle \rho(B), \{\}, \{\}, \{\}\rangle$\\
    \hline
  \end{tabular}
\end{table}

\noindent The first two rows of Table~\ref{tab:mergeinteractions} illustrate
the associativity of both merge operations. The third and fourth row show
the interaction between the different merges. Note that the asymmetry of the
projective merge operation results in an unbalanced behaviour in the fourth
row, where the projective merge retains precedence over the assertive merge
operation. Intuitively, this is because projected content has a status that
is prior to asserted content, which means that it must be allowed to come
first.  Finally, the fifth row illustrates the equivalence for the last
possible configuration. Here, again, the asymmetry of the projective merge
prevents an associative equivalence. Since the content of the first
PDRS~($A$) needs to be $\alpha$-converted on the basis of the label of the
second PDRS ($B$), we introduce an auxiliary PDRS via a function
$\mathcal{E}$ that creates an empty PDRS with the label of its argument
PDRS. With this auxiliary PDRS, the desired equivalence can be derived.

With the merge operations in place, we can combine basic (P)DRSs in order to
form larger meaning representations. In particular the distinction between
assertive and projective merge in PDRT provides a handle for constructing
meaning representations that consist of different types of content: asserted
content and projected content. In the next section, we define unresolved
structures on the basis of these merge operations, in order to represent
lexical meaning.

\begin{definition}[Resolvedness]~\\
A DRS $K$/PDRS $P$ is resolved iff:
\begin{itemize}
  \item $K$/$P$ does not contain any unresolved merges.
\end{itemize}
\end{definition}

\subsection{Unresolved Structures}

The operations described so far allow for the combination of \emph{complete}
structures, representing for example sentences or propositions, but there is
no way to build up structures from even smaller building blocks, such as
words, since the current framework is simply not powerful enough to express
the meaning of such constituents. For this, we need a way to define
unresolved structures that still need to be combined with some additional
content in order to form a complete PDRS.  Here, we follow
\citeasnoun{muskens1996combining}, who defined Compositional DRT,
a formalism that combines the dynamic semantics of DRT with a Montague style
composition procedure.

In Compositional DRT (CDRT), each syntactic category can be defined using an
unresolved DRS, defined by means of lambda abstractions. For example, a noun
is considered to introduce a condition on a referent that still needs to be
defined, which can be represented as follows (using the flat representation
for DRSs): 

\ex. ``\textit{man}'':~~ $\lambda x.$\flatdrs{}{man($x$)}

When combined with a suitable referent, this unresolved DRS is transformed
into a resolved DRS with a free variable, which can become bound after
merging with a DRS that contains this variable in its universe. So, the
interaction between beta reduction (the elimination of lambda-terms) and
merge operations is crucial for obtaining proper DRSs. In CDRT, this
interaction is explicitly part of the lexical semantics associated with the
syntactic structures, as the unresolved structure representing the
determiner ``\textit{a}'' illustrates: 

\ex. ``\textit{a}'':~~ 
  $\lambda p.\lambda q.(($\flatdrs{$x$}{}$~+~p(x))+q(x))$

A determiner is thus defined as introducing a DRS with a single referent,
which needs to be merged with two unresolved DRSs in order to form
a completely resolved structure (i.e., a sentence), or, in type-theoretic
terms, a determiner is of the type: $(e\rightarrow t)\rightarrow
(e\rightarrow t) \rightarrow t$. Now we can apply beta reduction to create
the noun phrase ``\textit{a man}'', which will be of type $(e\rightarrow t)
\rightarrow t$:

\ex. ``\textit{a man}'':~~ 
  $\lambda p.\lambda q.(($\flatdrs{$x$}{}$~+~p(x))+q(x))$ ($\lambda x.$
  \flatdrs{}{man($x$)}$)~\stackrel{\beta}{=}~$ $\lambda q.($
  \flatdrs{$x$}{man($x$)}$~+~q(x))$ 

Based on these type-theoretic principles, we can define unresolved
structures for a wide range of lexical items, ranging from simple structures
for basic items like nouns, to more complex abstractions for particles or
prepositions, for example, since these introduce complex dependencies.  In
the same way, a compositional version of PDRT can be defined, in which the
unresolved structures contain PDRSs that are combined using assertive and
projective merge. With this machinery, we can deal with projection in
a straightforward and intuitive way on the lexical level: presupposition
triggers introduce a projective merge to insert their content in their local
context, instead of introducing a constant
\citeaffixed{kamp1993discourse,muskens1996combining}{cf.} or requiring
a post-hoc projection mechanism \citeaffixed{sandt1992presupposition}{cf.}
(but note the correspondence to the lexical DRS semantics defined by
\citeasnoun{bos2003implementing} using the $\alpha$-operator).  For example,
a name like ``\emph{John}'' is a noun phrase that introduces
a presupposition about the existence of an entity named `John'; this can be
represented by means of a lambda-term of type $(e\rightarrow t)\rightarrow
t$, with a projective merge to combine the PDRS that introduces `John' with
the unresolved structure:

\ex. ``\textit{John}'':~~ 
  $\lambda p.($\flatpdrs{$1$}{$1\gets x$}{$1\gets$John($x$)}{}$~*~p(x))$

When combined with a term of type $e\rightarrow t$, the pointer associated
with `John' ($1$ in the example above) will become free because by
definition of projective merge it is not renamed to match the label of the
context PDRS (see Definition~\ref{def:pmerge}). Note, however, that the
presupposition may still become bound in the larger discourse, when part of
a PDRS that is combined with an antecedent PDRS that matches its label. This
implies that determining the projection site of a presupposition is part of
constructing its lexical semantics, which makes sense in the light of the
context-dependency of meaning. 

Table~\ref{tab:lexPDRS} shows the unresolved PDRSs for a set of basic
lexical items. The examples illustrate that the interchanging of assertive
and projective merge nicely captures the correspondence between projecting
and non-projecting sibling items, such as the determiners ``\textit{the}''
and ``\textit{a}'', and the factive verb ``\textit{know}'' and its
non-factive equivalent ``\textit{believe}''. In
section~\ref{sec:operations}, we will discuss some more (operations on)
lexical semantics, and in section~\ref{sec:playing} we will come back to the
issue of determining the right projection site for projected content.
%XXX Wel doen!

\begin{table}
  \caption{PDRT representations for a set of lexical items.}
  \label{tab:lexPDRS}
  \centering\small
\begin{tabular}{| l | c  | c |}
\hline
{\bf\normalsize Item}  & {\bf\normalsize Category} & 
  {\bf\normalsize PDRT Semantics}\\
\hline
\normalsize{a}       & DET & 
  $\lambda p.\lambda q.(($\pdrs{$1$}{$1\gets x$}{}{}$~+~p(x))+q(x))$\\
%\hline
\normalsize{the}     & DET & 
  $\lambda p.\lambda q.(($\pdrs{$1$}{$1\gets x$}{}{}$~+~p(x))*q(x))$\\
%\hline
\normalsize{dog}     & N  & 
  $\lambda x.$(\pdrs{$1$}{}{$1\gets$dog($x$)}{})\\
%\hline
\normalsize{John}    & NP & 
  $\lambda p.($\pdrs{$1$}{$1\gets x$}{$1\gets$John($x$)}{}$~*~p(x))$\\
%\hline
\normalsize{happy}   & ADJ &  
  $\lambda p. \lambda x.($\pdrs{$1$}{$1\gets e$}{$1\gets$happy($e$)\\
    $1\gets$Patient($e,x$)}{}$~+~p(x))$\\
%\hline
\normalsize{his}     & POS & 
  $\lambda p.\lambda q. \lambda r.((q(\lambda x.($\pdrs{$1$}{$1\gets y$}{
    $1\gets$male($x$)\\ $1\gets$of($y,x$)}{}$))~+~p(y))~*~r(y))$\\
%\hline
\normalsize{bark}    & V & 
  $\lambda p. \lambda q. (p(\lambda x. ($\pdrs{1}{$1\gets e$}{
    $1\gets$walk($e$)\\ $1\gets$Theme($e,x$)}{}$+~q(e))))$\\
%\hline
\normalsize{love}    & V & 
  $\lambda p. \lambda q.\lambda r. (q(\lambda x. (p(\lambda y.($\pdrs{1}{
    $1\gets e$}{$1\gets$love($e$)\\ $1\gets$Experiencer($e,x$)\\ 
    $1\gets$Stimulus($e,y$)}{}$+~r(e))))))$\\
%\hline
\normalsize{believe} & V & 
  $\lambda p. \lambda q. \lambda r. (q(\lambda x.($\pdrs{$1$}{$1\gets e$
    ~~$1\gets k$}{$1\gets$believe($e$)\\ $1\gets$Agent($e,x$)\\
    $1\gets$Theme($e,k$)\\ $1\gets k:(p(\lambda y.($\flatpdrs{$2$}{~}{~}{~}$))
    ~+~p(\lambda z.($\flatpdrs{$3$}{~}{~}{~}))}{}$~+~r(e))))$\\
%\hline
\normalsize{know}    & V & 
  $\lambda p. \lambda q. \lambda r. (q(\lambda x.($\pdrs{$1$}{$1\gets e$
    ~~$1\gets k$}{$1\gets$know($e$)\\ $1\gets$Agent($e,x$)\\
    $1\gets$Theme($e,k$)\\ $1\gets k:(p(\lambda y.($\flatpdrs{$2$}{~}{~}{~}$))
    ~*~p(\lambda z.($\flatpdrs{$3$}{~}{~}{~}))}{}$~+~r(e))))$\\
\hline
\end{tabular}
\end{table}
