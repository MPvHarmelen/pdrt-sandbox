\section{Background}\label{sec:background}

In the dynamic semantic framework of Discourse Representation Theory
\citeaffixed{kamp1993discourse}{DRT;}, the representation of the meaning
a discourse is constituted by means of recursive structures called DRSs,
which contain a set of discourse referents and a set of conditions on these
referents. DRSs are often visualized using a box-representation, with the
set of referents appearing in the top of the box, and the set of conditions
on these referents below.  Example \Next shows the DRS of a donkey sentence
\citeaffixed{geach1962reference}{cf.}, illustrating the embedded structure
introduced by the implication.

\ex. If a farmer owns a donkey, he feeds it.\\
\hspace*{-0.2cm}{\small \drs{}{\drs{x~~y}{farmer(x)\\ donkey(y)\\ own(x,y)}$\Rightarrow$\drs{}{feed(x,y)}}}

Each box can be seen as representing an (embedded) \emph{context},
reflecting the logical form of the discourse.  Crucially, the accessibility
between contexts determines anaphoric binding; in the example above, the
antecedent of the conditional is accessible from the consequent, and as
a result the variables introduced by the pronouns become bound by their
corresponding antecedents. Note that in this framework, form determines
interpretation, meaning that the context in which some content is introduced
determines how it is interpreted. In order to account for phenomena that do
not abide by this type of structural interpretation, such as presuppositions
and conventional implicatures \citeaffixed{potts2005logic}{cf.}, we have
proposed an extension to the DRT formalism, called Projective DRT
\citeaffixed{venhuizen2013iwcs,venhuizen2014salt}{PDRT;}. In PDRT, an
explicit distinction is made between the context in which some linguistic
content is \textit{introduced}, and where it is \textit{interpreted},
reflecting the separation between linguistic surface form and
truth-conditional interpretation.

The representation structures in Projective DRT are called PDRSs, and each
PDRS introduces a unique \emph{label}.  All referents and conditions of
a PDRS are associated with a \emph{pointer}, which is used to indicate in
which context the material is interpreted by means of binding the pointer to
a context label.  In the default case, the pointer is bound by the label of
its local PDRS, so content is interpreted locally, thus indicating asserted
material. In the case of projected content, however, the content should be
interpreted in some higher context; this is indicated by either binding the
pointer to an accessible PDRS label, or by means of a free pointer. Pointing
to a higher PDRS means that the content is interpreted within the context of
that PDRS. In case the pointer occurs free, no interpretation site has been
determined yet; the pointer may still become bound during discourse
construction, or it may remain free, which means that the content needs to
be accommodated. The context indicated by a free pointer needs to be
accessible from the context in which it is introduced.  To formalize this,
PDRSs are enhanced with a set of \MAPs: Minimally Accessible Projected
contexts, which are basically tuples describing pairs of subordinating
contexts. This additional level of information also allows for the
incorporation of other constrains between (projected) contexts, such as
those posed by embedded presuppositions. 

Example \Next[a] shows the PDRS of a sentence with only asserted content.
The labels introduced by the PDRSs are shown on top of each PDRS, and the
pointers associated with referents and conditions are indicated using the
$\gets$-operator. The \MAPs~can be shown below the conditions of a PDRS, as
illustrated in \Next[b], where the context indicated by the free pointer
needs to be accessible from the current context.

\begin{flushleft}
\begin{minipage}{0.85\linewidth}
\begin{multicols}{2}
\ex. \a. Nobody sees a man.\\
\hspace*{-0.3cm}{\small
  \pdrs{1}{}{
    $1\gets\neg$\pdrs{2}{$2\gets x, 2\gets y$}{
      $2\gets$person($x$)\\ $2\gets$man($y$)\\ $2\gets$see($x,y$)
    }{}
  }{}
}
\b. Nobody sees John.\\
\hspace*{-0.3cm}{\small
  \pdrs{1}{}{
    $1\gets\neg$\pdrs{2}{$2\gets x, 5\gets y$}{
      $2\gets$person($x$)\\ $5\gets$John($y$)\\ $2\gets$see($x,y$)
    }{$2 \leq 5$}
  }{}
}

\end{multicols}
\end{minipage}\\
\end{flushleft}

\noindent In \Last[a], all pointers are bound by their local PDRS, meaning
that all material is accommodated locally (note that we here assume
a non-specific reading of the indefinite ``a man'', in which it takes narrow
scope under the negation operator).  This representation is identical to the
standard DRT representation of this sentence, except for the addition of
labels to PDRSs and pointers to all referents and conditions. In \Last[b],
on the other hand, the proper name ``John'' triggers a presupposition about
the existence of someone called `John'. The pointer occurs free, meaning
that no antecedent has been found yet. In case no further context becomes
available, the interpretation site can be determined based on different
heuristics. For example, the pragmatic constraints proposed by
\citeasnoun{sandt1992presupposition} may be applied, resulting in a reading
where the presupposition is globally accommodated. 
%According to \citeasnoun{sandt1992presupposition} determining an
%appropriate accommodation site depends on several essentially pragmatic
%constraints, such as informativity and consistency.  Two important semantic
%(i.e., logical?) constraints are that the interpretation site is accessible
%from the introduction site, and that not variables become free after
%projection; these constraints are explicit in PDRS by means of the \MAPs.
%XXX account not dependent on this analysis


So, instead of moving semantic material on the representation level to the
context of interpretation, as in \citeasnoun{sandt1992presupposition} account,
projection is realized by setting a variable equation between pointers and
context labels.  This allows for a compositional treatment of projection
phenomena that does not assume a two-stage resolution algorithm.  Moreover,
the resulting semantic representation stays close to the linguistic
surface structure, making it easy to evaluate.  Only at the interpretation
stage, i.e., when providing a model-theoretic interpretation of the
representation structures, the content needs to be moved to its
accommodation site in order to obtain the desired interpretation.
%%%XXX



