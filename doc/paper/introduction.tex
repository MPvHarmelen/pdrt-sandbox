\section{Introduction}

Semantic frameworks aim to be wide coverage, as well as formally adequate.
DRT: one of the most influential.

Formally well-defined: model-theoretic backbone
\cite{kamp1981theory,kamp1993discourse}, compositional construction
\cite{muskens1996combining}.

Wide coverage: anaphora, tense \cite{kamp1981theory}, quantification \&
plurality \cite{kamp1993discourse}, attitude reports
\cite{asher1986belief,asher1989belief,zeevat1996neoclassical,maier2009presupposing}
discourse structure \cite{asher2003logics}, and presupposition
\cite{sandt1992presupposition,kamp1994drs,krahmer1998presupposition,geurts1999presuppositions}

We have proposed a novel treatment of presupposition in which it is treated
as part of a larger class of projection phenomena, does not require
a two-stage resolution strategy, and distinction between asserted and
projected content (i.e., explicit information structure): PDRT
\cite{venhuizen2013iwcs}.  Pointers indicate interpretation site. Can be
shown to account for various projection phenomena.  However, adding
projection pointers affects fundamental aspects of DRT non-trivially:
accessibility, binding, construction. 

In this paper, we will describe the formal implications of extending DRT
with projection pointers. After briefly describing the motivation behind
this DRT extension in Section~\ref{sec:background}, we systematically
derive the formal definitions of PDRT from the DRT definitions for building
and combining structures, and show where they diverge
(Sections~\ref{sec:building} and~\ref{sec:combining}).  By providing
a translation from PDRT to DRT in Section~\ref{sec:translating}, we show
that PDRT inherits all formal interpretive properties of DRT, together with
its inference mechanisms.  We introduce an implementation of DRT and PDRT,
called \textsc{pdrt-sandbox}, which implements all these definitions, and
provides a full-fledged toolkit for use in NLP applications. In
Section~\ref{sec:playing}, we offer some directions for applying the
formalism and implementation. Finally, Section~\ref{sec:conclusions}
concludes the paper.





%Presuppositions are a fundamental aspect of linguistic meaning since they
%relate the content of an utterance to the unfolding discourse. A definite
%description like ``the man'', for example, introduces a male discourse
%entity that is salient in the current discourse context.  Yet, in many
%formal semantic accounts, these phenomena have been treated as deviations
%from standard meaning construction, e.g., by resolving them only in
%a second step of processing \cite{sandt1992presupposition}. This is
%unsatisfactory for a number of reasons, mainly because it precludes an
%incremental compositional treatment of projection phenomena such as
%presuppositions.  Therefore, we have recently proposed Projective DRT
%(PDRT), which builds upon the widely used Discourse Representation Theory
%\cite{kamp1981theory,kamp1993discourse}.  PDRT deals with projected content
%during discourse construction, without losing any information about where
%information is introduced or interpreted \cite{venhuizen2013iwcs}.

%In order to evaluate its properties and applications, we implemented PDRT as
%a Haskell library, called \textsc{pdrt-sandbox}. This library includes machinery
%for representing PDRSs, as well as traditional DRSs, a translation from PDRT
%to DRT and FOL, and the different types of merge operations for (P)DRSs.
%Unresolved structures with lambda-abstractions \cite{muskens1996combining} are
%implemented as pure Haskell functions by exploiting Haskell's
%lambda-theoretic foundations. Several (P)DRS representations can be chosen
%as target output, including the well-known box-representations,
%a set-theoretic notation and flat, non-recursive, table-structures.
%\textsc{pdrt-sandbox} provides a comprehensive toolkit for use in NLP applications,
%and paves way for a more thorough understanding of the behaviour of
%projection phenomena in language.  
