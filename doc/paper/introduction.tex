\section{Introduction}\label{sec:introduction}

The semantic property of projection, traditionally associated with
presuppositions, has challenged many structure-driven formal semantic
analyses. Linguistic content is said to project if it is interpreted outside
the scope of an operator that syntactically subordinates it. This happens,
for example, if a presupposition-triggering expression is introduced within
the scope of a negation: in the sentence ``\textit{It is not the case that
the man is happy}'' the definite noun phrase ``\textit{the man}''
syntactically occurs inside the scope of the negation, but it is interpreted
as if occurring outside of its scope (corresponding to the reading: $\exists
x(man(x) \wedge \neg happy(x))$). This behaviour has often been treated as
a deviation from standard meaning construction, despite the prevalence of
expressions exhibiting it. Therefore, we have proposed a formalism that
centralizes the property of projection as a strategy for integrating
material into the foregoing context. The formalism is called Projective
Discourse Representation Theory \citeaffixed{venhuizen2013iwcs}{PDRT;}, and
is an extension of Discourse Representation Theory
\cite{kamp1981theory,kamp1993discourse}. In PDRT, presuppositions are
treated as part of a larger class of \textit{projection phenomena}
\citeaffixed{simons2010projects}{cf.}, including anaphora and conventional
implicatures \cite{potts2005logic}. By associating linguistic material with
a \textit{pointer} to indicate the interpretation site, an explicit
distinction is made between the surface form of an utterance, and its
logical interpretation. The analysis can be shown to account for various
projection phenomena \citeaffixed{venhuizen2014salt}{e.g., conventional
implicatures;}. Crucially, however, adding projection pointers to all
linguistic material affects the formal properties of DRT non-trivially; the
occurrence of projected material at the interpretation site results in
non-hierarchical variable binding, and violates the traditional DRT notion
of accessible contexts, thereby compromising the basic construction
procedure.

In this paper, we will describe the formal implications of extending DRT
with projection pointers. After briefly describing the motivation behind
this DRT extension, we systematically derive the formal definitions of PDRT
from the DRT definitions for building and combining structures, and show
where they diverge (Sections~\ref{sec:building} and~\ref{sec:combining}). By
providing a translation from PDRT to DRT in Section~\ref{sec:translating},
we show that PDRT inherits all formal interpretive properties of DRT,
together with its inference mechanisms.  We introduce an implementation of
DRT and PDRT, called \textsc{pdrt-sandbox}, which implements all these
definitions, and provides a full-fledged toolkit for use in NLP
applications. In Section~\ref{sec:playing}, we offer some directions for
applying the formalism and implementation. Finally,
Section~\ref{sec:conclusions} concludes the paper.

\subsection{Discourse Representation Theory}

In the dynamic semantic framework of Discourse Representation Theory
\citeaffixed{kamp1993discourse}{DRT;}, the meaning of a discourse is
represented by means of recursive structures called `Discourse
Representation Structures' (DRSs). A DRS consists of a set of discourse
referents and a set of conditions on these referents. Conditions may be
either \textit{basic}, reflecting a property or a relation between
referents, or \textit{complex}, reflecting logical structure introduced by
semantic operators such as negation, implication, and modal operators. DRSs
are often visualized using a box-representation, with the set of referents
appearing in the top of the box, and the set of conditions on these
referents below.  Example \Next shows the DRS of a donkey sentence
\citeaffixed{geach1962reference}{adapted from}. This example shows a DRS
with a single complex condition representing an implication, which itself
consists of two embedded DRSs, the first containing two referents and a set
of three basic conditions, and the second containing a single basic
condition.

\ex. If a farmer owns a donkey, he feeds it.\\\hspace*{-0.2cm}
  \drs{}{
    \drs{x~~y}{farmer(x)\\ donkey(y)\\ own(x,y)}
    $\Rightarrow$
    \drs{}{feed(x,y)}}

Each (embedded) DRS can be seen as representing an \textit{context},
together reflecting the logical form of the discourse. Crucially, anaphoric
binding of referents is determined on the basis of the accessibility between
contexts; in the example above, the antecedent DRS of the implication is
accessible from the consequent DRS, and as a result the variables introduced
by the pronouns become bound by the referents introduced in the antecedent
DRS. 

The straightforward analysis of anaphora was one of the main motivations
behind the development of DRT \cite{kamp1981theory,heim1982semantics}, but
it has been shown that the framework can account for a wide range of other
linguistic phenomena, including: tense \cite{kamp1981theory}, quantification
\& plurality \cite{kamp1993discourse}, attitude reports
\cite{asher1986belief,asher1989belief,zeevat1996neoclassical,maier2009presupposing}
discourse structure \cite{asher2003logics}, and presupposition
\cite{sandt1992presupposition,krahmer1998presupposition,geurts1999presuppositions}.
While illustrating the flexibility the framework, the analysis of
presuppositions in particular also emphasizes some limitations of the
framework. Next we will describe this analysis and motivate the proposed
extension to DRT, which was developed in particular to account for what we
call projection phenomena.

\subsection{Beyond the surface structure}

In the DRT framework \textit{form} determines \textit{interpretation}, that
is, the context in which some content is introduced determines how it is
interpreted. Thus, if some content occurs within the (syntactic) scope of,
for example, an implication or a negation in the linguistic surface form, it
will be interpreted within the logical scope of this operator. This tight
correspondence between form and meaning challenges a parsimonious account of
presuppositions and other \textit{projection phenomena}
\citeaffixed{potts2005logic}{e.g., conventional implicatures; cf.}, as these
are generally interpreted outside the logical scope of any operator that
embeds them \citeaffixed[for an overview]{simons2010projects}{see}.  That
is, replacing the indefinite noun phrase `a farmer' in example \Last by the
proper name `John' does not just imply replacing the predicate in the
associated DRS, but it requires moving the referent and condition describing
`John' to outside the scope of the implication in order to obtain the
appropriate interpretation: there exists some person named `John', for whom
it holds that if he owns a donkey, then he feeds it.  This process of
resolving presuppositions in DRT was formalized by
\citeasnoun{sandt1992presupposition}. In the account that has become known
as `binding and accommodation theory'
\cite{geurts1999presuppositions,beaver2002presupposition,bos2003implementing},
\citename{sandt1992presupposition} treats presupposition projection as
a variety of anaphora resolution: a presupposition first occurs marked at
its introduction site, and after discourse construction it is either
\textit{bound} to an accessible antecedent, or \textit{accommodated} at an
accessible context. This is illustrated in example \Next, where the
presupposition triggered by the proper name `John' yet needs to be resolved
in Stage I \Next[a], and has been accommodated to the global context in
Stage II \Next[b].

\begin{flushleft}
\begin{minipage}{0.85\linewidth}
\begin{multicols}{2}
\ex. If John owns a donkey, he feeds it.
\a.Stage I\\\hspace*{-0.2cm}{
  \drs{}{
    \drs{y}{donkey(y)\\ own(x,y)\\ \begin{tabular}{l}\noalign{\smallskip}
        \begin{tabular}{:l:} \hdashline x\\\hdashline John(x)\\\hdashline 
       \end{tabular}\smallskip\end{tabular}}
    $\Rightarrow$
    \drs{}{feed(x,y)}}
  }
  \b.\vspace*{0.1cm}Stage II\\\hspace*{-0.2cm}{
  \drs{x}{John(x)\\
    \drs{y}{donkey(y)\\ own(x,y)}
    $\Rightarrow$
    \drs{}{feed(x,y)}}
  }

\end{multicols}
\end{minipage}\\
\end{flushleft}

\noindent Although the algorithm proposed by binding and accommodation
theory results in representations with the desired interpretation for
presuppositions, the process of arriving at this interpretation, as well as
the obtained representations, are not completely satisfactory. Firstly, the
two-stage process of deriving the representations is at odds with
a compositional construction procedure for DRSs, as presuppositions are only
resolved \textit{after} discourse construction, and the intermediate
representations with unresolved DRSs are ill-defined with respect to formal
properties like accessibility and variable binding. Secondly, since in the
final representations no distinction can be made between asserted and
presupposed content, the information structure of the discourse becomes
obliterated \cite{kracht1994logic,krahmer1998presupposition}.

%, and distinction between asserted and
%projected content (i.e., explicit information structure)

In order to provide a parsimonious account of presuppositions, as part of
the larger class of projection phenomena, we have proposed an extension to
DRT, called Projective DRT \citeaffixed{venhuizen2013iwcs}{PDRT;}. In PDRT,
an explicit distinction is made between the context in which some linguistic
content is \textit{introduced}, and where it is \textit{interpreted},
reflecting the separation between linguistic surface form and
truth-conditional interpretation.

\subsection{Projective DRT}

The representation structures in Projective DRT are called PDRSs, and each
PDRS introduces a \textit{label} that can be used as an identifier
\citeaffixed{asher2003logics}{similar to the context identifiers implicitly
assumed by other DRT extensions, for example SDRT;}.  All referents and
conditions of a PDRS are associated with a \textit{pointer}, which is used
to indicate in which context the material is \textit{interpreted} by means
of binding the pointer to a context label.  In the default case, the pointer
is bound by the label of the PDRS in which the content is introduced, thus
indicating asserted material which is interpreted locally. In the case of
projected material, however, the content should be interpreted in some
higher context; this is indicated by either binding the pointer to an
accessible PDRS label, or by means of a free pointer. Pointing to a higher
PDRS means that the content is interpreted within the context of that PDRS.
In case the pointer occurs free, no interpretation site has been determined
yet; the pointer may still become bound during discourse construction, or it
may remain free, which means that the content needs to be accommodated. 
%%betere overgang
The context indicated by a free pointer needs to be accessible from the
context in which it is introduced.  To formalize this, PDRSs are enhanced
with a set of \MAPs: Minimally Accessible Projection contexts, which are
basically tuples describing pairs of subordinating contexts
\citeaffixed{reyle1993dealing,reyle1995reasoning}{cf.}. This additional
level of information also allows for the incorporation of other constrains
between (projected) contexts, such as those posed by embedded
presuppositions. 

Example \Next[a] shows the PDRS of a sentence with only asserted content,
and \Next[b] shows a PDRS with a presupposition.  The labels introduced by
the PDRSs are shown on top of each PDRS, and the pointers associated with
referents and conditions are indicated using the $\gets$-operator. The
\MAPs~are shown below the conditions of a PDRS, where the context indicated
by the free pointer needs to be accessible from the current context.

\begin{flushleft}
\begin{minipage}{0.85\linewidth}
\begin{multicols}{2}
\ex. \a. Nobody sees any man.\\\hspace*{-0.3cm}{
  \pdrs{1}{}{
    $1\gets\neg$\pdrs{2}{$2\gets x, 2\gets y$}{
      $2\gets$person($x$)\\ $2\gets$man($y$)\\ $2\gets$see($x,y$)
    }{}
  }{}
}
\b. Nobody sees John.\\\hspace*{-0.3cm}{
  \pdrs{1}{}{
    $1\gets\neg$\pdrs{2}{$2\gets x, 5\gets y$}{
      $2\gets$person($x$)\\ $5\gets$John($y$)\\ $2\gets$see($x,y$)
    }{$2 \leq 5$}
  }{}
}

\end{multicols}
\end{minipage}\\
\end{flushleft}

\noindent In \Last[a], all pointers are bound by their local PDRS, meaning
that all material is accommodated locally.  This representation is identical
to the standard DRT representation of this sentence, except for the addition
of labels to PDRSs and pointers to all referents and conditions. In
\Last[b], on the other hand, the proper name ``John'' triggers
a presupposition about the existence of someone called `John'. The pointer
occurs free, meaning that no antecedent has been found yet. In case no
further context becomes available, the interpretation site can be determined
based on different heuristics. For example, the pragmatic constraints
proposed by \citeasnoun{sandt1992presupposition} may be applied, resulting
in a reading where the presupposition is globally accommodated. 
%%%Explain!
%According to \citeasnoun{sandt1992presupposition} determining an
%appropriate accommodation site depends on several essentially pragmatic
%constraints, such as informativity and consistency.  Two important semantic
%(i.e., logical?) constraints are that the interpretation site is accessible
%from the introduction site, and that not variables become free after
%projection XXX account not dependent on this analysis

So, instead of physically moving semantic material on the representation
level to the context of interpretation, as in
\possessivecite{sandt1992presupposition} account, projection is realized by
setting a variable equation between pointers and context labels.  This
allows for a compositional treatment of projection phenomena that does not
assume a two-stage resolution algorithm.  Moreover, the resulting semantic
representation stays close to the linguistic surface structure, making it
easy to evaluate.  Only at the interpretation stage, i.e., when providing
a model-theoretic interpretation of the representation structures, the
content needs to be moved to its accommodation site in order to obtain the
desired interpretation.
%%%XXX




