\section{Introduction}

Presuppositions are a fundamental aspect of linguistic
meaning since they relate the content of an utterance to the unfolding
discourse. A definite description like ``the man'', for example, introduces
a male discourse entity that is salient in the current discourse context.
Yet, in many formal semantic accounts, these phenomena have been treated as
deviations from standard meaning construction, e.g., by resolving them only
in a second step of processing \cite{sandt1992presupposition}. This is unsatisfactory
for a number of reasons, mainly because it precludes an incremental
compositional treatment of projection phenomena such as presuppositions.
Therefore, we have recently proposed Projective DRT (PDRT), which builds
upon the widely used Discourse Representation Theory \cite{kamp1981theory,kamp1993discourse}.
PDRT deals with projected content during discourse construction, without
losing any information about where information is introduced or interpreted
\cite{venhuizen2013iwcs}.

In order to evaluate its properties and applications, we implemented PDRT as
a Haskell library, called \textsc{pdrt-sandbox}. This library includes machinery
for representing PDRSs, as well as traditional DRSs, a translation from PDRT
to DRT and FOL, and the different types of merge operations for (P)DRSs.
Unresolved structures with lambda-abstractions \cite{muskens1996combining} are
implemented as pure Haskell functions by exploiting Haskell's
lambda-theoretic foundations. Several (P)DRS representations can be chosen
as target output, including the well-known box-representations,
a set-theoretic notation and flat, non-recursive, table-structures.
\textsc{pdrt-sandbox} provides a comprehensive toolkit for use in NLP applications,
and paves way for a more thorough understanding of the behaviour of
projection phenomena in language.  

In this paper...
